\def\webDOI{http://dx.doi.org}
\def\Onera{ONERA}
\def\elsA{\emph{elsA}}
\def\FIGS{/Users/ericsavin/Documents/Cours/VKI-UQ-2018/SLIDES-2018}
\def\figc{/Users/ericsavin/Documents//Projects/UMRIDA/BC02-RAE2822/ALBATROS}
\def\figUMRIDA{/Users/ericsavin/Documents/Projects/UMRIDA/FIGS}
\def\logos{/Users/ericsavin/Documents/Latex/LOGOS}

\documentclass[10pt]{beamer}

\usepackage{mathrsfs}
\usepackage{bbm}
\usepackage{amsmath}
\usepackage{dsfont}
%\usepackage[pdftex, pdfborderstyle={/S/U/W 1}]{hyperref}
\usepackage{color}
\usepackage{slashbox}
\usepackage{stmaryrd}

\usepackage{subfigure}

\definecolor{coultitre}{rgb}{0.41,0.05,0.05}  % marron
\definecolor{fondtitre}{rgb}{0.80,0.80,0.80}  % gris
\definecolor{mybrownred}{HTML}{CB9090}
\definecolor{Brown}{HTML}{3B1701}

%============= blue grey ===================

\definecolor{mygrey}{HTML}{2F6F6F}      % title--th\definecolor{mygreyl}{HTML}{708090}
\definecolor{mytgrey}{HTML}{002244}    % upper left
\definecolor{mytgreya}{HTML}{3F5A6F}
\definecolor{mytgreyl}{HTML}{688A9F}
\definecolor{mytgreyll}{HTML}{EDF2F6}  % bandeau fond (clair) titre sous-titre


\definecolor{myblue}{HTML}{1122BB}
\definecolor{myblue2}{HTML}{2244EE}
\definecolor{myblue3}{HTML}{7799EE}
%============================================

\mode<presentation> {
    \usetheme{CambridgeUS}
  \setbeamercovered{transparent}
\setbeamercolor{frametitle}{fg=mytgrey,bg=mytgreyll}
\setbeamercolor{author in head/foot}{bg=mytgreya}
\setbeamercolor{section in head/foot}{bg=mytgrey}
\setbeamercolor{title in head/foot}{fg=white,bg=mytgreyl}
\setbeamercolor{title}{fg=mygrey}
\setbeamercolor{date in head/foot}{fg=mytgrey}
\setbeamercolor{item}{fg=mytgreyl}
 }

%=================================================

\usepackage{subeqnarray}
%
\def\ds{\displaystyle}
\def\vo{\vspace{1mm}}
\def\vt{\vspace{2mm}}
\def\vr{\vspace{3mm}}
\def\vf{\vspace{4mm}}
\def\vv{\vspace{5mm}}
\def\vs{\vspace{7mm}}
\def\ve{\vspace{11mm}}
\def\begit{\begin{itemize}}
\def\endit{\end{itemize}}
\def\begen{\begin{enumerate}}
\def\enden{\end{enumerate}}
\def\beas{\begin{eqnarray*}}
\def\eeas{\end{eqnarray*}}
\def\Ir{{\textrm I}}

\newcommand{\vbase}{\psi}
\newcommand{\xigj}{\xi}
\newcommand{\nominal}[1]{\underline{#1}}
\newcommand{\xig}{{\bf \xi}}
\newcommand{\dimrv}{d}
\newcommand{\torder}{t}
\newcommand{\nquad}{q}

\newcommand{\drag}{C_D}
\newcommand{\Drag}{C_D}
\newcommand{\Dragj}[1]{C_{D{#1}}}
\newcommand{\lift}{C_L}
\newcommand{\moment}{C_M}
\newcommand{\Pitch}{C_M}
\newcommand{\AoA}{\alpha}
\newcommand{\Bend}{U}
\newcommand{\Twist}{\phi}
\newcommand{\Mach}{M}
\newcommand{\iexp}{\mathrm{e}}
\newcommand{\esp}{{\mathbb E}}
\newcommand{\Rset}{{\mathbb R}}


\newcommand{\PDFN}{{\mathcal N}}
\newcommand{\PDFb}{\beta_{\mathrm I}}
\newcommand{\Proba}{{\mathcal P}}
\newcommand{\Qroba}{{\mathcal Q}}

\newcommand{\Cov}{\text{Cov}}
\newcommand{\Var}{\text{Var}}
\newcommand{\imax}{\text{max}}

%=============================================================================

\title
[ ]{ {\large M\'ethodes d'estimation d'erreur\\et propagation des incertitudes en CFD}}

\author
[\'E. Savin]{Jacques Peter$^1$, \underline{\'Eric Savin}$^{2,3}$}

\institute
[\Onera]
{\inst{1} \Onera\ -- D\'ept. A\'erodynamique, A\'ero\'elasticit\'e, Acoustique\\
\vskip3pt\inst{2} \Onera\ -- D\'ept. Traitement de l'Information \& Syst\`emes
\vskip3pt\inst{3} CentraleSup\'elec -- D\'ept. M\'ecanique \& G\'enie Civil}

\date{3 octobre 2019}


\AtBeginSection[
  {\frame<beamer>{\frametitle{Outline}
    \tableofcontents[currentsection]}}%
]%
{
  \frame<beamer>{
    \frametitle{Outline}
    \tableofcontents[currentsection]}
}


%%%%%%%%%%%%%%%%%%%%%%%%%%%%%%%%%%%%%%%%%%%%%%%%%%%%%%%%%%%%%%%%%%%%%%%%%%%%%%%%%%%%%%%%%%

\begin{document}

%====================================================================================

\begin{frame}
\titlepage
\vskip-10pt
\begin{center}
\includegraphics[scale=.15]{\logos/logo-f}
\end{center}
%\begin{figure}
%\includegraphics[height=1.4cm]{logo-onera-ident-quadri-HD}
%\end{figure}
\end{frame}

%\pgfdeclareimage[height=0.8cm]{institution-logo}{logo-onera-ident-quadri-HD}
%\logo{\pgfuseimage{institution-logo}} 

%=====================================================================================
%
\begin{frame}{Outline}
\tableofcontents
\end{frame}
%
%%%%%%%%%%%%%%%%%%%%%%%%%%%%%%%%%%%%%%%%%%%%%%%%%%%%%%%%%%%%%%%%%%%%%%%%%%%%%%%%%%%%%%

\section{Introduction. Need for Uncertainty Quantification (UQ)}

%%%%%%%%%%%%%%%%%%%%%%%%%%%%%%%%%%%%%%%%%%%%%%%%%%%%%%%%%%%%%%%%%%%%%%%%%%%%%%%%%%%%%%
%
%
\begin{frame}{Need for UQ} {Example I: drag evaluation}

\begit
\item Deterministic drag of airplane in cruise 
     \begit
     \item Total drag $\drag$ at cruise nominal Mach number ($\Mach$=0.82)  $\drag(0.82)$
     \item A/C shape satisfying constraints on lift, pitching moment, rolling moment...
     \endit
\vr
\item Actually cruise flight Mach number $\Mach$ varies:
    \begit
   \item Waiting for landing slot;
   \item Speeding up to cope with pilot maximum flight time;
   \endit
 \item[$\rightarrow$] Variable Mach number described by $D(\Mach)$.
\vr
\item Robust calculation of airplane cruise drag: compute $\int \drag(\Mach) D(\Mach) d\Mach$, instead of $\drag(0.82)$.

\endit
%
\end{frame} 

%=====================================================================================================

\begin{frame}{Need for UQ} {Example II: fan design}
%
\begit
\item Fan operational conditions subject to changes in wind conditions
\vt
\item Manufacturing subject to tolerances
\vt
\item Robust design accounts for:
  \begit
  \item variability of external parameters,
  \item tolerances for internal parameters.
  \endit
\endit
\begin{figure}[!h]
\begin{center}
\fbox{{\includegraphics[width=3.3cm]{\FIGS/Figures/Robust_Design.png}}}
\caption{Robust design \scriptsize{(from cenaero.be)}}
\end{center}
\end{figure}
%
\end{frame} 

%=====================================================================================================

\begin{frame}{Need for UQ} {Example III: validation process}
%
\begit
\item Unknown data in experiment
     \begit
     \item Upwind Mach number (equivalent to far-field Mach number in free-stream) not fully controlled in wind tunnels $d\Mach=0.001$     
     \endit
\vo
\item Unknown physical constant needed in numerical model
     \begit
     \item Wall roughness constant (milled, brazed, eroded surface...)
     \endit
\vo
\item Discrepancy in a computational/experimental validation process!
%
\vo
\item Compute the mean and standard deviation of the output(s) of interest due to the uncertain inputs
\endit
%
\end{frame} 
%
%****************************************************************************************
%
\begin{frame}{UQ inputs and outputs} {Definition of uncertain inputs}
%
\begit
\item UNCERTAINTY QUANTIFICATION: describes the stochastic behaviour of OUTPUTS of interest due to uncertain INPUTS
\vr
\item Overview of CFD actual uncertain INPUTS
  \begit
  \item Geometrical (manufacturing tolerance)
  \item Operational: flow at boundaries (far field, injection...)
  \item Reference: \scriptsize{T. P. Evans, P. Tattersall, J. J. Doherty: Identification and quantification of uncertainty sources in aircraft related CFD computations--An industrial perspective. In {\sl Proc. of NATO RTO-AVT-147 Symp. on Computational Uncertainty in Military Vehicle Design}, Paper \#6. Athens, Greece, Dec. 3-6, 2007}
%     {\footnotesize\sl Proceedings of RTO-MP-AVT-147 -- Evans T.P., Tattersall P. and
%      Doherty J.J.:~Identification and quantification of uncertainty
%     sources in aircraft related CFD-computations - An industrial
%     perspective. 2007. }
  \endit  
\vr
\item Stochastic behaviour of OUTPUTS
  \begit
  \item (Most often) mean and variance  
  \item Range = min and max possible values of outputs due to stochastic inputs
  \item Probability that an output exceeds a threshold
  \endit  
\endit
%
\end{frame} 
%
%===============================================================================================
%
\begin{frame}{Three issues with UQ} {1 -- Terminology}
%
\begit
%
\item Lack of agreement on the definition of ``error'', ``uncertainty''...
\vr
\item\href{\webDOI/10.2514/4.472855}{AIAA Guide G-077-1998}: \underline{Uncertainty} is a potential deficiency in any phase or activity of the modeling
 process that is due to the lack of knowledge. 
\underline{Error} is a recognizable deficiency in any phase or activity of the modeling process that is not due to the lack of knowledge.
\vr
\item \href{https://www.asme.org/codes-standards/find-codes-standards/v-v-20-standard-verification-validation-computational-fluid-dynamics-heat-transfer}{ASME Guide V\&V 20} (in its simpler version adopted for the Lisbon Workshops on CFD uncertainty):
  The \underline{validation comparison error}\footnote{\scriptsize{Verification: solving the equations right, validation: solving the right equations}} is defined as the difference between the simulation value and
 the experimental data value. It is split in numerical, model, input and data errors (assumed to be independant).  \underline{Numerical} (resp. input, model, data)
 \underline{uncertainty} is a bound of the absolute value of numerical (resp. input, model, data) \underline{error}.
%
\endit
%
\end{frame} 
%
%=****************************************************************************************
%
\begin{frame}{Three issues with UQ}{ 2 -- UQ validation and verification } 
%
\begit
\item UQ CFD-based exercise leads to standard deviation of some outputs
\vr
\item  Compare this standard deviation to the discretization error 
  \begit
  \item Richardson extrapolation method (1920's), Grid Convergence Index (GCI)...
  \item Adjoint based formulas for functional outputs\footnote{\href{\webDOI/10.1137/S0036144598349423}{\scriptsize{N. A. Pierce, M. B. Giles: Adjoint recovery of superconvergent functionals from PDE approximations. {\sl SIAM Rev.} {\bf 42}(1), 247-264 (2000)}}}$^,$\footnote{\href{\webDOI/10.1006/jcph.2000.6600}{\scriptsize{D. A. Venditti, D. L. Darmofal: Adjoint error estimation and grid adaptation for functional outputs: application to quasi-one-dimensional flow. {\sl J. Comput. Phys.} {\bf 164}(1), 204-227 (2000)}}}
   \endit
\vr
\item  Compare this standard deviation to the modeling error 
  \begit
  \item  Run several Reynolds-Averaged Navier-Stokes (RANS) models 
  \item  Run better models than RANS 
  \endit
\vr
\item  Numerical UQ investigation only makes sense if standard deviation due to uncertain inputs not much smaller
 than modelling or discretization error
\endit
%
\end{frame} 
%
%****************************************************************************************
%
\begin{frame}{Three issues with UQ}{3 -- Lack of shared well-defined problems?}
%
\begit
  \item Quite difficult to get information from industry in order to define relevant UQ exercises
  \vr 
  \item Quite difficult to understand when industry uses UQ and when industry uses multi-point analysis / optimization to deal with
   parameter variations 
  \vr 
  \item Do not only common problems with in-house CFD and chosen UQ method. Also share
    \begit  
    \item mathematical test cases with specific complexity
    \item mathematical test cases derived from  industrial cases (using surrogates)
    \endit 
  \item[] or it is difficult/impossible to split the influence of discrepancies in CFD methods and the one in UQ methods
\endit
%
\end{frame} 
%
%****************************************************************************************
%
\begin{frame}{Slides and lecture notes}
%
\begit
  \item \Onera\ involved in EU projects, RTO project on UQ 
  \vf 
  \item Provide accessible information for non-experts
  \item Examples, illustrations, explicit 2D formulas...
  \vf 
  \item Slides and lecture notes
\endit
%
\end{frame} 
%
%
%%%%%%%%%%%%%%%%%%%%%%%%%%%%%%%%%%%%%%%%%%%%%%%%%%%%%%%%%%%%%%%%%%%%%%%%%%%%%%%%%%%%%%

\section{Probability basics, Monte-Carlo, surrogate-based Monte-Carlo}

%%%%%%%%%%%%%%%%%%%%%%%%%%%%%%%%%%%%%%%%%%%%%%%%%%%%%%%%%%%%%%%%%%%%%%%%%%%%%%%%%%%%%%
%
\begin{frame}{Basics of probability (1)} 
%
\begit
\item[] A classical introduction to probability basics involves
%
\vr
\item event (one dice value, one Mach number value)
\vr
\item a sample space $\Omega$ (all six dice values, interval of Mach number values) 
\item set of events space ${\cal A}$ ($\sigma$-algebra)  set of subsets of $\Omega$, stable by union, intersection, complementation relative to $\Omega$, including null set $\emptyset$ and $\Omega$ 
\item a probability function $P$ on  ${\cal A}$ such that $ P(\Omega)=1$,$ P(\emptyset)=0$, plus natural properties for
 complementary parts and union of disjoint parts 
\vs
\item [OUT] random variables $X$ depending on the event $\xi$ (like $\smash{C_{Dp}}$ or $\smash{C_{Lp}}$ of an airfoil depending on the far-field Mach number through Navier-Stokes equations)
%
\endit
\end{frame} 
%
%**************************************************************************************************************
%
\begin{frame}{Basics of probability (2)} 
%
\begit
\item[] {\bf Discrete example: fair 6-face dice thrown once}
\vr
\item event  $\xi = 1,2,3,4,5$ or $6$;
\item sample space $\Omega =\{1,2,3,4,5,6\}$;
\item set of events ($\sigma$-algebra) ${\cal F}=2^\Omega$ = null set plus all discrete sets of these numbers $\{\emptyset,\{1\},\{2\},\{3\},\{4\},\{5\},\{6\},
   \{1,2\},\{1,3\},\{1,4\},\{1,5\},\{1,6\},$ $\{2,3\},...\{1,2,3,4,5,6\}\}$;
\item probability function $P$:~$P(\emptyset)= 0$, $P(\{1\})= 1/6$, $ P(\{2\})= 1/6$,... $P(\{1,2\})=1/3$, $P(\{1,3\})=1/3$, $P(\{1,4\})=1/3$,... $P(\{1,2,3,4,5,6\})=1$.
\vt 
\item random variables $X$, for example, dice value to the power three...
\endit
%
\end{frame} 
%
%**************************************************************************************************************
%
\begin{frame}{Basics of probability (3)} 

\begit
\item[] {\bf Continuous  example: far-field Mach number $\smash{\Mach_\infty}$ in [0.81,0.85]}
\vr
\item event $\xi$ = a Mach number value in [0.81,0.85]  
\item sample space $\Omega$ =  [0.81,0.85]
\item set of events ($\sigma$-algebra) ${\cal F}=2^\Omega$ = all subparts of [0.81,0.85]
\item probability function $P$. Probability  of (union of) intervals $I \in {\cal F}$
\item[] to be defined from a probability density function $D$,
 integrating $D$ over $I$.
%
\footnotesize{
 $$  \textsf{Example: }   ~~~~  D_\phi( \phi) = \frac{35}{32}(1-\phi^2)^3\,,\quad \phi \in~[-1,1]\,,\quad\phi = (\xi-0.83)/0.02 $$
 $$   D_\xi(\xi) = \frac{1}{0.02} D_\phi( \phi) = \frac{1}{0.02}\frac{35}{32}\left[1-\left(\frac{\xi-0.83}{0.02}\right)^2\right]^3 $$ }
\vt 
\normalsize
\item possible random variables $X$ = lift, drag, pitching moment of a wing... with variable Mach number $\smash{\Mach_\infty}$ (``event`` $\xi$) in the far-field
\endit
%
\end{frame} 
%
%**************************************************************************************************************

\begin{frame}{Basics of probability (4)}{Example of probability density functions}

Set of probability density functions of $\smash{\PDFb}$ distributions on $[0,1]$ with the $\alpha-1$ 
$\beta - 1$ convention for exponents
 $$ D_b(x)=\PDFb(x;\alpha,\beta)= \frac{ x^{\alpha-1}(1-x)^{\beta-1} }{\int_0^1 t^{\alpha-1}(1-t)^{\beta-1}dt }\,,\quad x \in [0,1] $$
%
\begin{figure}
\begin{center}
\includegraphics[angle=0,width=5.5cm]{\FIGS/Figures/Beta_distribution_pdf_0_1.png}
\end{center}
\end{figure}
%
\end{frame} 
%
%****************************************************************************************
%
\begin{frame}{Need for UQ} {Intrusive vs non-intrusive methods}
%
\begit
\item Non-intrusive methods. No change in the analysis code
 \begit
 \item Post-processing of deterministic simulations
 \endit
%
\vv
\item Intrusive methods.\footnote{\href{\webDOI/10.1007/978-1-4612-3094-6}{\scriptsize{R. Ghanem, P. D. Spanos: {\sl Stochastic Finite Elements: A Spectral Approach}. Springer,  New York NY (1991)}}}$^,$\footnote{\href{\webDOI/10.1007/978-90-481-3520-2}{\scriptsize{O. P. Le Ma\^{\i}tre, O. Knio: {\sl Spectral Methods for Uncertainty Quantification. With Applications to Computational Fluid Dynamics}. Springer, Dordrecht (2010)}}} Changes in the analysis code
  \begit
  \item Stochastic expansion of state/primitive variables
  \item Galerkin projections. Larger set of equations
  \vt
  \item Probably not feasible for large industrial codes
  \endit
\endit
\end{frame} 
%
%**************************************************************************************************************
%
\begin{frame}{ Monte-Carlo -- 1} 
%
\begit
\item[] {\bf Monte-Carlo mimics the law of the event  in a series of calculations}
%
\item[] Reference method for all uncertainty propagation methods
%
\vf
\item Generation of a $N$ samples $(\xi^1,\xi^2,\dots\xi^p,\dots\xi^N)$ of the PDF $D(\xi)$
%  \begit  
%   \item[] Example : sampling for normal distribution. $a^k$, $b^k$ indep. uniformly distributed in [0.,1.]
%      $\xi^k =\sqrt{-2~ln(a^k)}cos(2\pi b^k) $
%  \endit
\item Computation of corresponding flow fields $W(\xi^p), p \in \llbracket 1,N\rrbracket$
\item Computation of functional outputs  ${\cal J}(\xi^p)=J(W(\xi^p),X(\xi^p))$ 
\item Discrete unbiased estimations of mean and variance:
%
{\footnotesize
 $$ \esp({\cal J})= \int {\cal J}(\xi) D(\xi)d\xi \simeq {\color{blue}{\bar{{\cal J}_N} = \frac{1}{N}\sum_{p=1}^{N} {\cal J}(\xi^p)}}$$
%
 $$  \sigma^2_{\cal J} =
 \esp( ({\cal J}-\esp({\cal J}))^2)= \int ({\cal J}(\xi)-\esp({\cal J}))^2 D(\xi)d\xi \simeq  {\color{blue}{\sigma^2_{{\cal J}_N} = \frac{1}{N-1}\sum_{p=1}^{N} ({\cal J}(\xi^p)-\bar{{\cal J}_N})^2}}  $$}
%
\item Need to quantify accuracy of estimation
\endit
%
\end{frame} 
%%
%****************************************************************************************************************************************************
%
\begin{frame}{Monte-Carlo -- 2 }{Accuracy of mean} 

\begit
\item Scalar case, variance $\sigma_{\cal J}$ is known, $N$ sampling size, $\sqrt{N} \frac{\bar{{\cal J}_N}-\esp({\cal J})}{\sigma_{\cal J}} \leadsto \PDFN(0,1)$ (normal distribution) 
\vt
 \item  Probability density function (PDF) of $\PDFN(0,1)$: $D_n(x)=\frac{1}{\sqrt{2\pi}} \iexp^{-\frac{x^2}{2}}$
 \item  Symmetric cumulative distribution function (CDF): $\Phi_n(x)=\frac{1}{\sqrt{2\pi}}\int_{-x}^x \iexp^{-\frac{t^2}{2}}dt$
%  
\item With $\epsilon$ confidence:
 $$ \esp({\cal J}) \in \left[ \bar{{\cal J}_N}-u_{\epsilon}\frac{\sigma_{{\cal J}}}{\sqrt{N}},
                   \bar{{\cal J}_N}+u_{\epsilon}\frac{\sigma_{{\cal J}}}{\sqrt{N}} \right]\,, ~~
   \epsilon = \frac{1}{\sqrt{2\pi}}\displaystyle\int_{-u_\epsilon}^{u_\epsilon} \iexp^{-\frac{t^2}{2}}dt$$

\begin{figure}
\begin{center}
\begin{tabular}{|c|c|c|c|c|}
\hline
$\epsilon$ & 0.5 & 0.9 & 0.95 & 0.99\\
\hline
$u_{\epsilon}$ & 0.674 & 1.645 & 1.960 & 2.576 \\
\hline
\end{tabular}
\end{center}
\end{figure}
%
\vt
\item {\bf Example}: with 99\% confidence
%   
  $$ \esp({\cal J}) \in\left[ \bar{{\cal J}_N}-2.576\frac{\sigma_{{\cal J}}}{\sqrt{N}},
                   \bar{{\cal J}_N}+2.576\frac{\sigma_{{\cal J}}}{\sqrt{N}} \right]\quad
   \left(0.99=\frac{1}{\sqrt{2\pi}}\displaystyle\int_{-2.576}^{2.576} \iexp^{-\frac{t^2}{2}}dt\right) $$
\endit
%
\end{frame} 
%
%********************************************************************************************************
%
\begin{frame}{Monte-Carlo -- 3}{Accuracy of mean} 
%
\begit
\item Scalar~case,~variance~$\sigma_{\cal J}$~is~\textcolor{blue}{unknown},~$N$~sampling~size,~$\sqrt{N}\frac{\bar{{\cal J}_N}-\esp({\cal J})}{\sigma_{{\cal J}_N}} \leadsto \mathcal{S}(N-1)$ (Student distribution)
\item With $\epsilon$ confidence: 
%
   $$ \esp({\cal J}) \in \left[ \bar{{\cal J}_N}-u_{\epsilon_{N-1}}\frac{\sigma_{{\cal J}_N}}{\sqrt{N}},
                    \bar{{\cal J}_N}+u_{\epsilon_{N-1}}\frac{\sigma_{{\cal J}_N}}{\sqrt{N}} \right]$$
% 
\item $u_{\epsilon_N}$ as function of $\epsilon$ and $N$ found in tables.  $u_{\epsilon_N}$ decreases with $N$ increasing
\item Student distribution converges to normal distribution for large $N$
\item Tables for $u_{\epsilon_{N-1}}$
\begin{table}
\begin{center}
\begin{tabular}{|c|c|c|c|c|c|}
\hline
\backslashbox{$\epsilon$}{$N-1$} & 1 & 2 & 20 & 30 & $\infty$ \\
\hline
0.95 & 12.71 & 4.303 & 2.086 & 2.042 & 1.960 \\ 
\hline
0.99 & 63.66 & 9.925 & 2.845 & 2.750 & 2.576 \\ 
\hline
\end{tabular}
\caption{Value of $u_{\epsilon_{N-1}}$ for Student distribution $ \mathcal{S}(N-1)$, $N \geq 2$}
%
\end{center}
\end{table}
\endit
%
\end{frame} 
% 
%********************************************************************************************************
%
%
\begin{frame}{Monte-Carlo -- 4 }{Accuracy of mean} 
%
\begit
\item Scalar~case,~variance~$\sigma_{\cal J}$~is~unknown,~$N$~sampling~size,~$\sqrt{N}\frac{\bar{{\cal J}_N}-\esp({\cal J})}{\sigma_{{\cal J}_N}} \leadsto \mathcal{S}(N-1)$ (Student distribution)
\vt
 \item  Student distribution\footnote{The random variable $T=\smash{\sqrt{\frac{N}{U}}}G\sim\mathcal{S}(N)$ where $G\sim\PDFN(0,1)$ and $U\sim\smash{\chi^2_N}$ independent from $G$ follows the $\smash{\chi^2_N}$ law with $N$ degrees of freedom ($N\geq 1$): $U=\smash{\sum_{n=1}^N G_n^2}$ where $G_n\sim\PDFN(0,1)$ independent identically distributed (i.i.d.).} $ \mathcal{S}(N)$ PDF: 
%
$$ D_{{\cal S}(N)}(x)=\frac{\Gamma(\frac{N+1}{2})}{\sqrt{N\pi}\,\Gamma(\frac{N}{2})} \left(1+\frac{x^2}{N}\right)^{-\frac{N+1}{2}}\quad\left(\Gamma(u)= \int_0^{+\infty} t^{u-1}\iexp^{-t}dt\right)$$
\vt
\item With $\epsilon$ confidence ($\epsilon \in ]0,1[$): 
  $$ \esp({\cal J}) \in \left[ \bar{{\cal J}_N}-u_{\epsilon_{N-1}}\frac{\sigma_{{\cal J}_N}}{\sqrt{N}},
                       \bar{{\cal J}_N}+u_{\epsilon_{N-1}}\frac{\sigma_{{\cal J}_N}}{\sqrt{N}} \right]
\quad\left(\epsilon = \displaystyle\int_{-u_{\epsilon_{N-1}}}^{u_{\epsilon_{N-1}}} D_{{\cal S}(N-1)}(t)dt\right) $$
\endit
%
\end{frame} 
%x
%********************************************************************************************************
%
%
\begin{frame}{Monte-Carlo -- 5}{Accuracy of estimation: variance (1) [skp'd]} 
%
\begit
\item Scalar case: mean $\esp({\cal J})$ is known
%
\item Estimation of variance 
  $$\sigma_{{\cal J}_N}^2=\frac{1}{N}\ds\sum_{p=1}^{N} ({\cal J}(\xi^p)-\esp({\cal J}))^2$$
%
\item Chi-square $\chi^2_N$ probability distribution defined on $[0,\infty[$ with PDF:
    $$D_{\chi^2_N}(x)=\displaystyle\frac{x^{\frac{N}{2}-1}\iexp^{-\frac{x}{2}}}{2^\frac{N}{2}\Gamma(\frac{N}{2})}\,,\quad x\in\Rset_+$$
%
\item Chi-square CDF:
     $$\Phi_{\chi^2_N}(x) = \displaystyle\int_0^x D_{\chi^2_N}(t) dt$$
%
\item Stochastic variable $$ N \frac{\sigma_{{\cal J}_N}^2}{\sigma_{{\cal J}}^2} \leadsto \chi^2_N$$
\endit
%
\end{frame} 
%
%******************************************************************************************************
%
\begin{frame}{Monte-Carlo -- 6}{Chi-square probabilistic density functions $D_{\chi^2_N}$ and cumulative density functions $\Phi_{\chi^2_N}$ [skp'd]} 
%
%
\begin{figure}
\includegraphics[width=6.cm]{\FIGS/Figures/Chi-square_distributionPDF}
\includegraphics[width=6.cm]{\FIGS/Figures/Chi-square_distributionCDF}
\end{figure}
%
\end{frame}
%
%********************************************************************************************************
%
\begin{frame}{Monte-Carlo -- 7}{Accuracy of variance (2) [skp'd]} 

\begit
\item Scalar case: mean $\esp({\cal J})$ is known, $ N \frac{\sigma_{{\cal J}_N}^2}{\sigma_{{\cal J}}^2} \leadsto \chi^2_N$

\item With $\epsilon=1-\alpha$ confidence:
 $$ \Phi_{\chi^2_N}^{-1}\left(\frac{\alpha}{2}\right)  \leq N \frac{\sigma_{{\cal J}_N}^2}{\sigma_{{\cal J}}^2} \leq \Phi^{-1}_{\chi^2_N} \left(1-\frac{\alpha}{2}\right)$$

\item With $\epsilon=1-\alpha$ confidence:
  $$\sigma_{{\cal J}}^2 \in \left[ N\frac{\sigma_{{\cal J}_N}^2}{\Phi_{\chi^2_N}^{-1}(1-\frac{\alpha}{2})},N\frac{\sigma_{{\cal J}_N}^2}{\Phi^{-1}_{\chi^2_N}(\frac{\alpha}{2})}\right]$$
\endit
%
\begin{table}
\begin{center}
\begin{tabular}{|c|c|c|c|}
\hline
\backslashbox{$\epsilon$}{$N$} &  2 & 20 & 30 \\
\hline
0.005  & 10.597 & 39.997 & 53.672 \\ 
\hline
0.995  & 0.0100 & 7.434  & 13.787 \\ 
\hline
\end{tabular}
\caption{Value of $\Phi^{-1}_{\chi^2_N}(\epsilon)$}
\end{center}
\end{table}
%
\end{frame} 
%
%********************************************************************************************************
%
\begin{frame}{Monte-Carlo -- 8 }{Accuracy of variance (3) [skp'd]}
%
\begit
\item Application. With 99\% confidence, depending on $N$ number of samples
$$
\begin{array}{ll}
N=2 & \sigma^2_{{\cal J}} \in [0.189, 200]\times \sigma^2_{{\cal J}_2} \\
N=20 & \sigma^2_{{\cal J}} \in [0.500, 2.69]\times\sigma^2_{{\cal J}_{20}}\\
N=30 & \sigma^2_{{\cal J}} \in [0.559,2.18]\times\sigma^2_{{\cal J}_{30}} \\
N=100 & \sigma^2_{{\cal J}} \in [0.713,1.49]\times\sigma^2_{{\cal J}_{100}}
\end{array}
$$
%\vt
\item Convergence speed of bounds towards 1.
\item The cumulative distribution of the Chi-Square law $\Phi_N(x)$ can be expressed as
  $$ \Phi_{\chi^2_N}(x) = \frac{1}{\Gamma (N/2)}\displaystyle\int_{0}^{x/2} t^{N/2}\iexp^{-t}dt = \frac{\gamma(N/2,x/2)}{\Gamma(N/2)}$$
where $\gamma$ is the lower incomplete $\Gamma$ function\footnote{\scriptsize{$\gamma(s,x)=\int_0^x t^{s-1}\iexp^{-t}dt$}}
%\vf
\item Check properties of (the inverse of) $\Phi_{\chi^2_N}$ 
\item Check convergence speed of  $N/\Phi_{\chi^2_N}^{-1} (1-\frac{\alpha}{2})$ and $N/\Phi_{\chi^2_N}^{-1}(\frac{\alpha}{2})$
\endit
%
\end{frame} 
%
%********************************************************************************************************
%
\begin{frame}{Monte-Carlo -- 9}{Accuracy of  variance (4) [skp'd]} 
%
\begit
%
\item Scalar case: mean $\esp({\cal J})$ is unknown, stochastic variable $(N-1) \frac{\sigma_{{\cal J}_N}^2}{\sigma_{{\cal J}}^2} \leadsto \chi^2_{N-1}$ 
%
\item With $\epsilon=1-\alpha$ confidence:
%
 $$\sigma_{{\cal J}}^2 \in \left[ (N-1)\frac{\sigma_{{\cal J}_N}^2}{\Phi^{-1}_{\chi^2_{N-1}}(1-\frac{\alpha}{2})},
                                             (N-1)\frac{\sigma_{{\cal J}_N}^2}{\Phi^{-1}_{\chi^2_{N-1}}(\frac{\alpha}{2})}\right]$$
%
\endit
%
\begin{table}
\begin{center}
\begin{tabular}{|c|c|c|c|c|}
\hline
\backslashbox{$\epsilon$}{$N$} & 3 & 4 & 20 & 30 \\
\hline
0.005 & 10.597 & 12.838 & 38.582 & 52.336 \\ 
\hline
0.995 & 0.0100 & 0.0717 & 6.844  & 13.121 \\ 
\hline
\end{tabular}
\caption{Value of $\Phi^{-1}_{\chi^2_{N-1}}(\epsilon)$}
\end{center}
\end{table}
%
\end{frame} 
%
%********************************************************************************************************
%
\begin{frame}{Monte-Carlo -- 10 }{ Cost issue. Regularity of outputs} 
%
\begit
%
\item Typical realistic estimation of accuracy of mean estimated by Monte-Carlo is: with a $N$ point sampling, with $99\%$ confidence,
  $$ \esp({\cal J}) \in \left[ \bar{{\cal J}_N}-u_{0.99,{N-1}}\frac{\sigma_{{\cal J}_N}}{\sqrt{N}},
                       \bar{{\cal J}_N}+u_{0.99,{N-1}}\frac{\sigma_{{\cal J}_N}}{\sqrt{N}} \right]$$
%
 where $u_{0.99,1}=63.66$, $u_{0.99,2}=9.925$, $u_{0.99,3}=5.841$, $u_{0.99,9}=3.250$, $u_{0.99,19}=2.861$, $u_{0.99,99}=2.756$...
 decreasing with the number of samples $N$ towards limiting value $2.576...$
%
\item Convergence speed of Monte-Carlo for mean value estimation is $\frac{1}{\sqrt{N}}$ 
%
\item Increasing precision of Monte-Carlo estimation by a factor of 10 requires multiplying the number
 of evaluations by a factor of 100
\vt
\item[] {\bf Extremely expensive if one evaluation requires numerical solution of Euler or RANS equations}
\vt
\item ... But Monte-Carlo is independent of the number of random inputs!
%
\endit
%
\end{frame} 
%
%********************************************************************************************************
%
\begin{frame}{Monte-Carlo -- 11}{ Cost issue. Regularity of outputs} 
%
\begit
%
\item Convergence speed of Monte-Carlo for mean value estimation is $\frac{1}{\sqrt{N}}$ 
\vo
\item[] {\bf Extremely expensive if one evaluation requires numerical solution of Euler or RANS equations}
\vf
\item Besides outputs of CFD calculations are often very regular functions of the parameters of interest
\vt
\item[] {\bf Take advantage of the regularity of (random) output variables seen as function of (stochastic/events) inputs variables }
\vt
\item Derive a surrogate of the output variables as function of the input variables using {\bf specific stochastic surrogates}
 $\rightarrow$ next section
\vt
\item Derive a surrogate of the output variables as function of the input variables using {\bf general surrogates}
 $\rightarrow$ end of this section
\vt
\item Calculate mean, variance, kurtosis, range, risk... for the surrogate
%
\endit 
%
\end{frame} 
%\vr
%********************************************************************************************************
%
\begin{frame}{Meta-model based Monte-Carlo} 

\begin{figure}
\begin{center}
\includegraphics[width=11cm]{\FIGS/Figures/mc_mm2.pdf}
\caption{Monte-Carlo method with meta-models}
\end{center}
\end{figure}

\end{frame} 

%===========================================================================================


\begin{frame}{Meta-models} 

\begit
\item Restriction: approximation of a function of interest. What kind of surrogate can be used?
\vs
\item[1] Classical metamodels: Kriging, Radial Basis Function, Support Vector Regression (used regularly at \Onera\footnote{\href{https://tel.archives-ouvertes.fr/tel-00771799}{{\scriptsize M. Bompard: {\sl Mod\`eles de substitution pour l'optimisation globale de forme en a\'erodynamique et m\'ethode locale d'optimisation sans param\'etrisation}. PhD Thesis, Universit\'e Nice -- Sophia Antipolis, December 2011}}}).
\vt
\item[2] Other meta-models of specific interest for UQ: generalized polynomial Chaos (gPC), Stochastic Collocation (SC)
\vt
\item[3] Other model of specific interest for large dimensions: adjoint based linear or quadratic Taylor expansion
\vs
\item Influence of meta-model accuracy on mean and variance accuracy?
\endit


\end{frame} 

%===========================================================================================

\begin{frame}{Application of metamodel-based Monte-Carlo} 


\begit
\item Confidence intervals on lift $\lift$ with uncertainty on $AoA$
\item Nominal configuration: NACA0012, $M=0.73$, $Re=6M$, $AoA=3^{\circ}$
\item \Onera\ \elsA\ code\footnote{\href{\webDOI/10.1051/meca/2013056}{{\scriptsize L. Cambier, S. Heib, S. Plot: The \Onera\ \elsA\ CFD software: input from research and feedback from industry. {\sl Mechanics \& Industry} {\bf 14}(3), 159-174 (2013)}}} 
\item (RANS + $k$--$\omega$ Wilcox turbulence model) solver (Roe flux + Van Albada lim.)
\endit

\begin{figure}[!h]
\begin{center}
\includegraphics[width=6.5cm]{\FIGS/Figures/profile.pdf}
\caption{Mesh}
\end{center}
\end{figure}

%
\end{frame} 
%
%===========================================================================================


\begin{frame}{Distribution of uncertainty} 

\begit
\item Beta distribution (parameters $(\alpha,\beta)=(3,3)$) over $[-1,1]$ 
%
$$ D_b(\xi) =\frac{15}{16}(1-\xi)^2(1+\xi)^2=\frac{15}{16}(1-\xi^2)^2$$
%
\item PDF of angle of attack $AoA$ over $[2.9^\circ,3.1^\circ]$
 $$ D_a(AoA) =10 D_b(10(AoA-3)) $$ 
\endit
%
\begin{figure}[!h]
\begin{center}
\includegraphics[angle=270,width=7.cm]{\FIGS/Figures/beta2.png}
\caption{Beta distribution of $AoA$}
\end{center}
\end{figure}
%
\end{frame} 
%
%===========================================================================================
%
\begin{frame}{Monte-Carlo method for $\lift$ mean} 
%
\begin{figure}[!h]
\begin{center}
\fbox{{\includegraphics[angle=270,width=8cm]{\FIGS/Figures/classic_CL_mean.pdf}}}
\caption{Mean of $\lift$ coefficient and confidence interval}
\end{center}
\end{figure}
%
\end{frame} 
%
%===========================================================================================
%
\begin{frame}{Monte-Carlo method for $\lift$ variance} 
%
\begin{figure}[!h]
\begin{center}
\fbox{{\includegraphics[angle=270,width=8cm]{\FIGS/Figures/classic_CL_var.pdf}}}
\caption{Variance of $\lift$ coefficient and confidence interval}
\end{center}
\end{figure}
%
\end{frame} 
%
%===========================================================================================
%
\begin{frame}{Metamodel based Monte-Carlo: learning sample} 
%
\begit
\item Use learning sample based on roots of Chebyshev polynomials
\endit
%
\begin{figure}[!h]
\begin{center}
\fbox{{\includegraphics[angle=270,width=7cm]{\FIGS/Figures/tchebychev11.pdf}}}
\caption{Chebyshev distribution (11 points)}
\end{center}
\end{figure}
%
\end{frame} 
%
%===========================================================================================
%
\begin{frame}{Metamodel-based Monte-Carlo: reconstruction of $\lift$} 
%
\begin{figure}[!h]
\begin{center}
\fbox{{\includegraphics[width=8cm]{\FIGS/Figures/CL_AoA.png}}}
\caption{$AoA\mapsto m\lift(AoA)$}
\end{center}
\end{figure}
%
\end{frame} 
%
%===========================================================================================
%
\begin{frame}{Metamodel-based Monte-Carlo for $\lift$ mean }
{Calling metamodel instead of CFD code} 
%
\begin{figure}[!h]
\begin{center}
\fbox{{\includegraphics[angle=270,width=8cm]{\FIGS/Figures/mm_CL_mean.pdf}}}
\caption{Mean of $m\lift$ coefficient and confidence interval}
\end{center}
\end{figure}
%
\end{frame} 
%
%===========================================================================================
%
\begin{frame}{Metamodel-based Monte-Carlo for $\lift$ variance} 
{Calling metamodel instead of CFD code} 
%
\begin{figure}[!h]
\begin{center}
\fbox{{\includegraphics[angle=270,width=8cm]{\FIGS/Figures/mm_CL_var.pdf}}}
\caption{Variance of $m\lift$ coefficient and confidence interval}
\end{center}
\end{figure}
%
\end{frame} 
%
%%%%%%%%%%%%%%%%%%%%%%%%%%%%%%%%%%%%%%%%%%%%%%%%%%%%%%%%%%%%%%%%%%%%%%%%%%%%%%%%%%%%%%

\section{Non-intrusive polynomial methods for 1D / tensorial $n$D propagation}

%%%%%%%%%%%%%%%%%%%%%%%%%%%%%%%%%%%%%%%%%%%%%%%%%%%%%%%%%%%%%%%%%%%%%%%%%%%%%%%%%%%%%%
%
\begin{frame}{Two polynomial methods for UQ. 1D and $n$D tensorial}
%
\footnotesize{
\begit
\item {\bf Stochastic specific polynomial surrogates}

\item For all non-intrusive methods
\vt
  \begit
  \item Presentation for one uncertain parameter $\xi$, probability density function $D(\xi)$
  \vo
  \item Extention to a vector of  two uncertain parameters ${\bf \xi}=(\xi_1,\xi_2)$ under the restriction that
 $$ D({\bf \xi})=D_1(\xi_1)\times D_2(\xi_2) $$
  and no sparsity is sought for = extension of $N$-point evaluation  method in 1D uses $N^2$ evaluations in dimension 2
  \vr
  \item Extrapolation to $d$-D to discuss complexity and cost 
  \endit
\vs
\item {\bf Generalized polynomial chaos method}
\vr
\item {\bf Stochastic collocation method}
\endit }
%
\end{frame}
%
%***************************************************************************************************
%
\begin{frame}{Generalized Polynomial Chaos (gPC) method -- 1}

\footnotesize{
\begit
\item Polynomial expansion of the quantity of interest, scalar output or vector 
%
 $$ F(\xi) \simeq gF(\xi) = \sum_{l=0}^{M} C_l P_l(\xi)$$
%
\item Coefficients of the expansion computable by different methods (quadrature,
 collocation)
\item Polynomial basis orthogonal for the dot product defined by the PDF $D(\xi)$
  $$  <P_l,P_m> = \int P_l(\xi) P_m(\xi) D(\xi) d\xi = \delta_{lm}$$
\item Straightforward calculation of mean and variance of the polynomial expansion (that approximates the
 quantity of interest)
\vs
\item Orthogonal polynomials: \href{http://people.math.sfu.ca/~cbm/aands/}{M. Abramowitz, I. A. Stegun: \emph{Handbook of Mathematical Functions}, chap. 22. National Bureau of Standards, Washington DC (1972)}.
\item Spectral expansions: \href{https://store.doverpublications.com/0486411834.html}{J. P. Boyd: \emph{Chebyshev and Fourier Spectral Methods (2nd ed.)}. Dover, Mineola NY (2001)}. 
\endit
}
%
\end{frame}
%
%***************************************************************************************************
%
\begin{frame}{Generalized Polynomial Chaos (gPC) method -- 2}{Families of orthogonal polynomials}

\footnotesize{ \begit
% 
 \item Normal distribution $\smash{D_n(\xi)=\frac{\iexp^{-\frac{\xi^2}{2}}}{\sqrt{2\pi}}}$ on $\Rset$ $\rightarrow$ (Probabilists') Hermite polynomials
 \vt
 \item Gamma distribution $\smash{D_g(\xi) = \iexp^{-\xi}}$ on $\smash{\Rset^+}$ $\rightarrow$ Laguerre polynomials
 \vt
 \item Uniform distribution $\smash{D_u(\xi)}=0.5$ on $[-1,1]$ $\rightarrow$ Legendre polynomials
 \vt
 \item Chebyshev distribution $\smash{D_{cf}(\xi)=\smash{\frac{1}{\pi\sqrt{1-\xi^2}}}}$ on $[-1,1]$ $\rightarrow$
 Chebyshev (first-kind) polynomials 
 \vt
 \item Chebyshev distribution $\smash{D_{cs}(\xi)=\sqrt{1-\xi^2}}$ on $[-1,1]$ $\rightarrow$
 Chebyshev (second-kind) polynomials 
 \vt
 \item Beta distribution $D_b(\xi) = \smash{\frac{1}{C_{\alpha\beta}}(1-\xi)^\alpha(1+\xi)^\beta}$ on $[-1,1]$ with $ -1<\alpha$, $-1<\beta$, and normalization constant $\smash{C_{\alpha\beta}}=\smash{\int_{-1}^1(1-u)^\alpha(1+u)^\beta du}$ $\rightarrow$ Jacobi polynomials (incl. Chebyshev and Legendre polynomials)
 \vr
 \item Non-usual probabilistic density functions $D_l(\xi)$ computed
 by Gram-Schmidt orthogonalization process.
%
\endit  }
%
\end{frame}
%
%*********************************************************************************************************
%
\begin{frame}{Generalized Polynomial Chaos (gPC) method -- 3}{Families of orthogonal polynomials}

\footnotesize{
\begit
% 
 \item Example: Stochastic variable in $\Rset$. Hermite polynomials for normal law $D_n(\xi)=\frac{\iexp^{-\frac{\xi^2}{2}}}{\sqrt{2\pi}}$ 
\vo
%
 \item First polynomials
 \begit
  \item $\overline{PH}_0(\xi)=1 $
  \item $\overline{PH}_1(\xi)=\xi$
  \item $\overline{PH}_2(\xi)=\xi^2-1$
  \item $\overline{PH}_3(\xi)=\xi^3-3\xi$
  \item $\overline{PH}_4(\xi)=\xi^4-6\xi^2+3$
 \endit
  \item Recursive definition 
  \begit
     \item  $\overline{PH}_0(\xi)=1$, $\quad\overline{PH}_1(\xi)=\xi$, $\quad\overline{PH}_{n+1}(\xi) = \xi \overline{PH}_n(\xi) -n \overline{PH}_{n-1}(\xi)$
   \endit 
 \vt
 \item Normalization $ PH_j(\xi)=\frac{1}{\sqrt{ j!}} \overline{PH}_j(\xi) $
 \vo
 \item Orthonormality relation for $PH$
 $$<PH_j,PH_k>=\int_\Rset PH_j(\xi)PH_k(\xi)D_n(\xi)d\xi = \delta_{jk}$$
%
\endit
}
%
\end{frame}
%
%**********************************************************************************************************************************
%
\begin{frame}{Generalized Polynomial Chaos (gPC) method -- 4}{Families of orthogonal polynomials}

\footnotesize{
\begit
 \item Example: Stochastic variable in [-1,1].  First-kind Chebyshev polynomials for probability density function
 $D_{cf}(\xi)=\frac{1}{\pi} \frac{1}{\sqrt{1-\xi^2}} $
% 
 \item Family of orthonormal polynomials for $<f,g>=\int_{-1}^{1}f(t)g(t)D_{cf}(t)dt $
     \begit
     \item $\overline{T}_0(\xi)=1 $
     \item $\overline{T}_1(\xi)=\xi$
     \item $\overline{T}_2(\xi)=2\xi^2-1$
     \item $\overline{T}_3(\xi)=4\xi^3-3\xi$
%     \item $\overline{T}_4(\xi)=8\xi^4-8\xi^2+1$
     \endit
 \item Recursive definition
    \begit
    \item  $\overline{T}_0(\xi)=1$, $\quad\overline{T}_1(\xi)=\xi$, $\quad\overline{T}_{n+1}(\xi) = 2\xi \overline{T}_{n}(\xi) - \overline{T}_{n-1}(\xi)$
    \endit
 \vo
 \item Normalization $ T_0 = \overline{T}_0$, $T_1=\sqrt{2}~ \overline{T}_1$... $T_n=\sqrt{2}~ \overline{T}_n$ $(n \geq 1)$
 \vo
 \item Orthonormality of the $T_j$
 $$<T_j,T_k> = \int_{-1}^{1}T_j(\xi)T_k(\xi)D_{cf}(\xi)d\xi = \delta_{jk}$$
 %\vt
 \item Specific property $\overline{T}_n(\cos\theta)=\cos(n\theta)$ (hence $\vert\vert \overline{T}_n \vert \vert_\infty \leq 1$) 
\endit
}
\end{frame}
%
%*****************************************************************************************************
%
\begin{frame}{Generalized Polynomial Chaos (gPC) method -- 5}{Polynomial expansion }

\footnotesize{
\begit

\item Expansion of a functional output depending on stochastic variable $\xi$ 
 $$ F(\xi) \simeq gF(\xi) = \sum_{l=0}^{M} C_l P_l(\xi)$$
\item Expansion of a field on part of the mesh depending on stochastic variable $\xi$ 
 ($i$ is a generic index for a part of the mesh nodes like wall nodes)
 $$ W(i,\xi) \simeq gW(i,\xi) = \sum_{l=0}^{M} C_l(i) P_l(\xi)$$
%
\item Accuracy of ideal $gW$ depending on degree and regularity. Theory of spectral expansions   
 %Accuracy of actual $gW$ expansion --  
\vf
\item Stochastic post-processing for $gW$ ($gF$) instead of $W$ ($F$) 
\item Straightforward calculation of $gW$ ($gF$) mean and variance 

\endit
}
\end{frame} 

%*********************************************************************************************************

\begin{frame}{Generalized Polynomial Chaos (gPC) method -- 6}{Coefficients computation (1/4) -- Gaussian quadrature}

\begit
%
\footnotesize{
\item Expansion of part of flow field  depending on stochastic variable $\xi$ and generic mesh index $i$
%
    $$ W(i,\xi) \simeq gW(i,\xi) = \sum_{l=0}^{M} C_l(i) P_l(\xi)$$
%
\item From orthonormality property $ C_l(i) =  <gW(i),P_l>$.  Under regularity assumptions $ C_l(i) =  <W(i),P_l>$  
}
\vf
\item  Proof: 
\scriptsize{ Assume $D$ is defined on an interval of $\Rset$ and bounded. Assume  uniform convergence of spectral expansion over its domain of definition 
    $$ W(i,\xi) = \sum_{l=0}^{\infty} C_l(i) P_l(\xi)$$
%
\item[] Multiply by  $P_n(\xi) D(\xi)$
     $$ W(i,\xi)P_n(\xi) D(\xi)  = \sum_{l=0}^{\infty} C_l(i) P_l(\xi) P_n(\xi) D(\xi) $$
%
\item[] Integrating over domain of definition of $D(\xi)$ yields
     $ C_n(i) =  <W(i),P_n> $ 
%
}
\endit

\end{frame} 
%
%*********************************************************************************************************

\begin{frame}{Generalized Polynomial Chaos (gPC) method -- 7}{Coefficients computation (2/4) -- Gaussian quadrature}
%
\footnotesize{
%
\begit
%
\item Expansion of part of flow field  depending on stochastic variable $\xi$ and generic mesh index $i$
%
    $$     gW(i,\xi) = \sum_{l=0}^M C_l(i) P_l(\xi)\,, \quad C_l(i) =  <gW(i),P_l> $$
%
\item Gaussian quadrature for
  $$ C_l(i)  = <W,P_l> =  \displaystyle\int W(i,\xi)P_l(\xi) D(\xi) d\xi$$
\item Computation by Gaussian quadrature associated to PDF $D$ with $g$ points. Exact integration of polynomials up to degree $2g-1$
\item Example of criteria for definition of number of points $g$ = enough points to recover orthogonality property at discrete level 
 for all polynomials of the expansions 
 $$ 2 M \leq 2g -1 $$    
%
\endit
%
}
%
\end{frame} 
%
%***************************************************************************************************
%
\begin{frame}{Generalized Polynomial Chaos (gPC) method -- 8}{Coefficients computation (3/4) -- Gaussian quadrature}
%
\footnotesize{
%
\begit
%
\item Expansion of part of flow field  depending on stochastic variable $\xi$ and generic mesh index $i$
%
    $$  gW(i,\xi) = \sum_{l=0}^M C_l(i) P_l(\xi)\,,\quad C_l(i) =  <gW(i),P_l> $$
%
\item $g$-point Gaussian quadrature associated to $D$
      $$ \int h(\xi) D(\xi) d\xi \simeq \sum_{k=1}^{g} \omega_k h(\xi_k) $$ 
  ($\omega_k$,$\xi_k$) depend on $D(\xi)$. Exact for polynomials up to degree $2g-1$
%
\vt
\item Calculation of gPC coefficients
  $$ C_l(i) = <W,P_l> =  \displaystyle\int W(i,\xi)P_l(\xi) D(\xi) d\xi = \sum_{k=1}^{g} \omega_k W(i,\xi_k)P_l(\xi_k)  $$
  $C_l(i)$ exact if $W(i,\xi)P_l(\xi)$ polynomial of $\xi$ of degree lower or equal to $2g-1$
%
\endit
%
}
%
\end{frame} 

%%*******************************************************************************************************

\begin{frame}{Generalized Polynomial Chaos (gPC) method -- 9}{Coefficients computation (4/4) -- collocation}
%
\footnotesize{
%
\begit

\item Other way: collocation or least-square collocation 
\item NB: less accuracy results than for Gauss quadrature
 \vf
\item Identify $W(i,\xi_l)$ and  $gW(i,\xi_l)$ for $M+1$ values of $\xi$.  Identify $F(\xi_l)$ and  $gF(\xi_l)$ for $M+1$ values of $\xi$. 

  $$  \sum_{l=0}^{M} C_l P_l(\xi_k) = F(\xi_k)\quad\forall k\in \llbracket 1,M+1\rrbracket \quad\textsf{solved for}\;C_l$$

\vt
\item[1] Number of $F$ evaluations $=$ number of coefficients. Linear system 
\item[2] Number of $F$ evaluations $>$ number of coefficients. Solve least-square problem problem
\item[3] Number of $F$ evaluations $<$ number of coefficients. See later ``sparsity-of-effects'' \& ``compressed sensing''
\vv
\item Matrix notation $ {\bf F}  $ column vector of $F$ values, ${\bf C}$ column vector of unknown polynomial coefficients 
 ${\bf K}$ matrix $K_{kl}=  P_l(\xi_k) $
%
    $$   {\bf K}{\bf C} = {\bf F}  $$
%
\endit
%
}
%
\end{frame} 
%
%************************************************************************************************************
%
\begin{frame}{Generalized Polynomial Chaos (gPC) method -- 10}{ Stochastic post-processing (1/3) }
%
\footnotesize
{
 $$ F(\xi) \simeq gF(\xi) = \sum_{l=0}^{M} C_l P_l(\xi)$$
%
\begit
\item Stochastic post-processing (mean and variance) done for the expansion $gF$ instead of $F$
   \begit 
\footnotesize{
   \item straightforward evaluation of mean value
      $$ \esp(gF(\xi)) = \int \left( \sum_{l=0}^{M} C_l P_l(\xi) \right)  D(\xi) d\xi = C_0 $$
  \item straightforward evaluation of variance
      $$ \esp((gF(\xi)-C_0)^2)=\int \left( \sum_{l=1}^{M}  C_l P_l(\xi) \right) ^2 D(\xi)d\xi = \sum_{l=1}^{M}  C_l^2 $$ 
 }
   \endit
\endit
%
}
\end{frame} 
%************************************************************************************************************
%
\begin{frame}{Generalized Polynomial Chaos (gPC) method -- 11}{ Stochastic post-processing (2/3) }
%
\footnotesize
{
 $$ F(\xi) \simeq gF(\xi) = \sum_{l=0}^{M} C_l P_l(\xi)$$
%
\begit
\item Stochastic post-processing (mean and variance) done for the expansion $gF$ instead of $F$
   \begit 
   \item Skewness 
      $$ \esp\left( \left(\frac{ gF(\xi) -\mu }{\sigma}\right)^3\right) = \frac{1}{ ( \sum_{l=1}^{M} C_l^2)^{3/2}}\int \left( \sum_{l=1}^{M} C_l P_l(\xi) \right)^3 D(\xi) d\xi  $$
 requires the knowledge/calculation of 
        $\int P_l(\xi) P_n(\xi) P_p(\xi) D(\xi) d\xi$ integrals\footnote{\href{\webDOI/10.1002/nme.5505}{\scriptsize{\'E.~Savin, B. Faverjon: Computation of higher-order moments of generalized polynomial chaos expansions. {\sl Int. J. Num. Methods Engng.} {\bf 111}(12), 1192-1200 (2017)}. Codes: \href{https://github.com/ericsavin/LinCoef}{\textsf{https://github.com/ericsavin/LinCoef}}}}
\vt
  \item Calculation of range. Sample $\xi$ and evaluate $gF(\xi)$  
\vt
  \item Probability of that $F$ exceeds a threshold $T$. Sample $\xi$ and evaluate $gF(\xi)$ for 
     $$\int  1_{\{gF(\xi)>T\}} D(\xi) d\xi$$  
   \endit
\endit
%
}
%
\end{frame} 
%
%%*******************************************************************************************************
%
\begin{frame}{Generalized Polynomial Chaos (gPC) method -- 12}{ Stochastic post-processing (3/3) }

\footnotesize
{
    $$   gW(i,\xi) = \sum_{l=0}^{M} C_l(i) P_l(\xi)$$
\begit
\item For vectors as well, stochastic post-processing (mean and variance) done for the expansion $gW$ instead of $W$
%
   \begit
   \item straightforward evaluation of mean value
      $$\esp(gW(i,\xi)) = \int \left( \sum_{l=0}^{M} C_l(i) P_l(\xi) \right) D(\xi) d\xi = C_0(i) $$
   \item straightforward evaluation of variance
      $$\esp((gW(i,\xi)-C_0(i))^2)=\int \left( \sum_{l=1}^{M}  C_l(i) P_l(\xi) \right) ^2 D(\xi)d\xi = \sum_{l=1}^{M}  C_l(i)^2 $$
   \item Estimation of skewness, kurtosis... 
   \item Estimation of range
   \item Estimation of probability to exceed a threshold 
   \endit
%
\endit
%
}
%
\end{frame} 
%
%**************************************************************************************************************
%
\begin{frame}{2D tensorial extension of gPC method -- 1}{ Definition }
%
\footnotesize{
\begit
%
\item 2 uncertain parameters  $(\xi_1,\xi_2) \in$ \Ir$^1 \times$ \Ir$^2 $ 
%
                   $$ D(\xi_1,\xi_2) = D^{\alpha}(\xi_1)D^{\beta}(\xi_2)$$ 
\vo
\item Families of orthogonal polynomials for  $D^\alpha(\xi_1)$ and $D^\beta(\xi_2)$ are $(P^\alpha_0,P^\alpha_1,P^\alpha_2,\dots)$
 and $(P^\beta_0,P^\beta_1,P^\beta_2,\dots)$
%
\vr
\item Polynomial extension (output functional case)
% 
     $$ F(\xi_1,\xi_2) \simeq gF(\xi_1,\xi_2) = \sum_{k\leq M^1,l \leq M^2} C_{kl} P^\alpha_k (\xi_1)  P^\beta_l (\xi_2)$$
%
\endit  }
%
\end{frame}
%
%***************************************************************************************************************
%
\begin{frame}{2D tensorial extension of gPC method -- 2}{ Tensorial product of two quadrature rules}
%
\footnotesize{
%
\begit
%
\item Calculate the $  (M^1+1)\times (M^2+1)$ coefficients by integration over interval \Ir$^1 \times$ \Ir$^2$ as 
%
    $$ C_{kl} = \int_{\Ir^1\times \Ir^2}  F(\xi_1, \xi_2)  P^\alpha_k (\xi_1)  P^\beta_l (\xi_2) D^\alpha(\xi_1)D^\beta(\xi_2) d\xi_1 d\xi_2 $$
%
\item Tensorial approach. First define the  tensorial product of two 1D Gaussian rules for integration in directions $\xi_1$ and $\xi_2$
 over \Ir$^1$ and \Ir$^2$ 
%
      $$  A[f] = \sum_{k=1}^{k=g^\alpha} \omega^\alpha_k f(\xi^\alpha_k) ~~~~~\left( \textsf{approximating }  \int_{\Ir^1 }f(u) D^\alpha(u) du \right) $$ ~~~~ 
      $$  B[g] = \sum_{l=1}^{l=g^\beta} \omega^\beta_l g(\xi^\beta_l) ~~~~~\left( \textsf{approximating }   \int_ {\Ir^2} g(v) D^\beta(v) dv \right ) $$
%
\item Tensorial quadrature  $A \otimes B$ over \Ir$^1 \times$ \Ir$^2$
        $$ A \otimes B[h] = \sum_{k\leq g^\alpha, l \leq g^\beta} \omega^\alpha_k \omega^\beta_l h(\xi^\alpha_k, \xi^\beta_l) $$
%
\endit
%
}
%
\end{frame}
%
%***************************************************************************************************************
%
\begin{frame}{2D tensorial extension of gPC method -- 3}{ Tensorial product of two quadrature rules}
%
\footnotesize{
%
\begit
%
\item Calculate the $  (M^1+1)\times (M^2+1)$ coefficients by integration over interval \Ir$^1 \times$ \Ir$^2$ as 
%
    $$ C_{kl} = \int_{\Ir^1\times \Ir^2}  F(\xi_1, \xi_2)  P^\alpha_k (\xi_1)  P^\beta_l (\xi_2) D^\alpha(\xi_1)D^\beta(\xi_2) d\xi_1 d\xi_2 $$
%
\vr
\item Tensorial quadrature  $A \otimes B$ over \Ir$^1 \times$ \Ir$^2$
        $$ A \otimes B[h] = \sum_{k\leq g^\alpha, l \leq g^\beta} \omega^\alpha_k \omega^\beta_l h(\xi^\alpha_k, \xi^\beta_l) $$
 (that is exact for $ \xi_ 1^p \xi_2^q$ if $ p \leq 2 g^\alpha-1 $ and $q \leq 2g^\beta-1 $) 
%
\vv
\item Calculation of gPC coefficients
%
   $$ C_{kl} = \int_{\Ir^1\times \Ir^2 } F(\xi_1, \xi_2) D^\alpha (\xi^1) D^\beta(\xi^2) d\xi_1 d\xi_2 \simeq  A \otimes B[F]
    =  \sum_{k\leq g^\alpha, l \leq g^\beta} \omega^\alpha_k \omega^\beta_l F(\xi^\alpha_k, \xi^\beta_l)  $$
%

\endit
%
}
%
\end{frame}
%
%***************************************************************************************************************
%
\begin{frame}{2D tensorial extension of gPC method -- 4}{Calculation of coefficients using the tensor product of two quadrature rules}
%
\footnotesize{
%
\begit
%
\item Calculate the $  M^1\times M^2$ coefficients by integration over $\Ir^1 \times \Ir^2$ as 
%
    $$ C_{kl} = \int  F(\xi_1,\xi_2)  P^\alpha_k (\xi_1)  P^\beta_l (\xi_2) D^1(\xi_1) D^\beta(\xi_2) d\xi_1 d\xi_2 $$
%
\item[] by tensorial quadrature rule 
% 
   $$ \int_{\Ir^1\times \Ir^2 } F(\xi_1, \xi_2) D^\alpha (\xi^1) D^\beta(\xi^2) d\xi_1 d\xi_2 \simeq   \sum_{k\leq g^\alpha, l \leq g^\beta} \omega^\alpha_k \omega^\beta_l F(\xi^\alpha_k, \xi^\beta_l)   $$
%
\vv
\item Requires $g^\alpha \times g^\beta$ flow calculations and evaluations of $F$
\endit
%
}
%
\end{frame}
%
%***************************************************************************************************************
%
\begin{frame}{2D tensorial extension of gPC method -- 5}{Calculation of coefficients using collocation}
%
\footnotesize{
%
\begit
%
\item Calculate the $(M^1+1)\times (M^2+1)$ coefficients of function expansion  
%
     $$ F(\xi_1,\xi_2) \simeq gF(\xi_1,\xi_2) = \sum_{k\leq M^1,l \leq M^2} C_{kl} P^\alpha_k (\xi_1)  P^\beta_l (\xi_2)$$
%
\item[]  by collocation by identifying the spectral expansion for 
 $(M^1+1) \times (M^2+1) $ points with exact evaluations
% 
     $$ \sum_{k\leq M^1,l \leq M^2} C_{kl} P^\alpha_k (\xi_1^s)  P^\beta_l (\xi_2^s) = F(\xi_1^s, \xi_2^s)\,,\quad s \in \llbracket 1,2,3,\dots(M^1+1) \times (M^2+1)\rrbracket$$
%
\item Use least square approach if more sampling points than coefficients
%   
\endit
%
}
%
\end{frame}
%
%***************************************************************************************************************
%
\begin{frame}{2D tensorial extension of gPC method -- 6}{Stochastic post processing}
%
\footnotesize{
%
\begit
%
\item gPC 2D expansion
%
     $$ gF(\xi_1,\xi_2) = \sum_{k\leq M^1,l \leq M^2} C_{kl} P^\alpha_k (\xi_1)  P^\beta_l (\xi_2)$$
%
\vt
\item Calculation of mean
      $$ \esp(gF) = \int \left( \sum_{k\leq M^1,l \leq M^2} C_{kl} P^\alpha_k (\xi_1)  P^\beta_l (\xi_2) \right) d\xi_1 d\xi_2 = C_{00} $$
\vt
\item straightforward evaluation of variance
\scriptsize{
 \beas    
 \Var(gF) &=& \esp((gF-C_{00})^2)\\
             &=&\int \left(  \sum_{k\leq M^1,l \leq M^2 } C_{kl} P^\alpha_k (\xi_1)  P^\beta_l (\xi_2) D(\xi_1,\xi_2)d\xi_1 d\xi_2 -C_{00}\right) ^2 D(\xi_1)^\alpha D^\beta(\xi_2)d\xi_1 d\xi_2 \\ 
             &=&\int \left(   \sum_{1\leq k\leq M^1,1\leq l \leq M^2} C_{kl} P^\alpha_k (\xi_1)  P^\beta_l (\xi_2) \right) ^2 D(\xi_1)^\alpha D(\xi_2)^\beta d\xi_1 d\xi_2 \\
             &=&\sum_{1\leq k\leq M^1,1\leq l \leq M^2} C_{kl}^2  \\ 
  \eeas
}
%
\vt
\item Calculation of variance
%   
\endit
%
}
%
\end{frame}
%
%********************* stochastic collocation is my dream in the night ********************************************
%
\begin{frame}{Stochastic collocation (SC) method -- 1}{Definition}
%
\footnotesize{
%
\begit
%
\item Another approach for non-intrusive polynomial chaos based on Lagrangian polynomial expansion.\footnote{\href{\webDOI/10.1029/97JD01654}{\scriptsize{M. A. Tatang, W. Pan, R. G. Prinn, G. J. McRae: An efficient method for parametric uncertainty analysis of numerical geophysical models. {\sl J. Geophys. Res.} {\bf 102}(D18), 21925-21931 (1997)}}}$^,$\footnote{\href{\webDOI/10.1137/040615201}{\scriptsize{D. Xiu, J. S. Hesthaven: High-order collocation methods for differential equations with random inputs. {\sl SIAM J. Sci. Comput.} {\bf 27}(3), 1118-1139 (2005)}}}$^,$\footnote{\href{\webDOI/10.2514/6.2007-317}{\scriptsize{G. J. A. Loeven, J. A. S. Witteveen, H. Bijl: Probabilistic collocation: An efficient non-intrusive approach for arbitrarily distributed parametric uncertainties. AIAA Paper \#2007-0317 (2007)}}}

\item Dedicated stochastic polynomial expansion using Lagrangian polynomials (sum of polynomials of degree $M$)
  $$ W(i,\xi) \simeq scW(i,\xi) = \sum_{l=1}^{M+1} W_l(i) H_l(\xi)\,, \quad H_l(\xi) = \prod_{m=1, m\neq l}^{M+1} \displaystyle\frac{(\xi-\xi_m)}{(\xi_l-\xi_m)} $$

\item  Note that $scW(i,\xi_l)=  \sum_{l=1}^{M+1} W_l(i) H_l(\xi_l) = W_l(i)$ $\rightarrow$ no coefficient calculation step
\item Compute flows (and extract part of state variables fields) $W(i,\xi_l)$ corresponding to the $\xi_l$ and substitute $W(i,\xi_l)$ to $W_l(i)$ 
%
 $$  scW(i,\xi) = \sum_{l=1}^{M+1} W(i,\xi_l) H_l(\xi) $$
\endit
%
}
\end{frame} 
%
%************************************************************************************************************
%
\begin{frame}{Stochastic collocation (SC) method -- 2}{Suitable set of points}
%
\footnotesize{
%
\begit
%
\item Polynomial expansion using Lagrangian polynomials
  $$ W(i,\xi) \simeq scW(i,\xi) = \sum_{l=1}^{M+1} W(i,\xi_l) H_l(\xi) $$
%
\item Definition of $(\xi_1, \xi_2,...\xi_{M+1})$? 
\vo
\item[1] $M+1$ points of the $(M+1)$-point Gaussian quadrature associated to $D(\xi)$ (most often, not absolutely necessary)
\item[2] Any set of $(M+1)$ distinct points
  \vo
  \begit
  \item Calculate mean and variance using the  $(M+1)$-point Gaussian quadrature associated to $D(\xi)$. Exact mean and variance. Not so simple formulas
  \item Calculate mean and variance using interpolating quadrature associated to the nodes. Inexact mean and variance.
  \endit 
%
\endit
%
}
%
\end{frame} 
%
%***************************************************************************************************
%
\begin{frame}{Stochastic collocation (SC) method -- 3}{Mean and variance evaluation 1/3} 

\scriptsize{
\begit
\item Stochastic post-processing (mean and variance) done for the expansion $scW$ instead of $W$
 -- In case the interpolation points $(\xi_1, \xi_2,\dots\xi_{M+1})$ are the $M+1$ points of the $(M+1)$-point Gauss quadrature associated to $D(\xi)$,
 the weights being $(\omega_1, \omega_2,\dots\omega_{M+1})$ 
\vr
\item straightforward evaluation of mean value (degree $M$ polynomial)
     $$ \esp(scW(i,\xi)) = \int scW(i,\xi) D(\xi) d\xi= \sum_{m=1}^{M+1} \omega_m scW(i,\xi_m) = \sum_{m=1}^{M+1} \omega_m W(i,\xi_m)  $$
%
\item straightforward evaluation of variance (degree $2M$ polynomial)
 \beas
     \esp((scW(i,\xi)-\esp(scW(i,\xi)))^2)&=&   \esp(scW(i,\xi)^2) -\esp(scW(i,\xi))^2 \\
                 &=&   \int scW(i,\xi)^2 D(\xi)d\xi  -\esp(scW(i,\xi))^2      \\
                 &=&   \sum_{m=1}^{M+1} \omega_m  scW(i,\xi_m)^2  -\esp(scW(i,\xi))^2\\
                 &=&  \sum_{m=1}^{M+1}  \omega_m W(i,\xi_m)^2- \left(\sum_{m=1}^{M+1} \omega_m W(i, \xi_m)\right)^2
\eeas
   \item Both exact from quadrature polynomial exactness. 
\endit
}
%
\end{frame} 
%
%***************************************************************************************************
%
\begin{frame}{Stochastic collocation (SC) method -- 4}{Mean and variance evaluation 2/3} 

\scriptsize{
\begit
\item Stochastic post-processing (mean and variance) done for the expansion $scW$ instead of $W$ -- In case the interpolation points $(\xi_1, \xi_2,\dots\xi_{M+1})$ are not the $M+1$ points of the $(M+1)$-point quadrature associated to $D(\xi)$. Note these Gauss quadrature points $(\nu_1, \nu_2,\dots\nu_{M+1})$ and the weights $(\omega_1, \omega_2,\dots\omega_{M+1})$ (no flow have been calculated for the $\nu_m$)
\item This quadrature is used for evaluations of mean and variance
\vr
\item Evaluation of mean value (degree $M$ polynomial)
     $$ \esp(scW(i,\xi)) = \int scW(i,\xi) D(\xi) d\xi = \sum_{m=1}^{M+1} \omega_m scW(i,\nu_m) $$
%
\item Evaluation of variance (degree $2M$ polynomial)
\beas
     \esp((scW(i,\xi)-\esp(scW(i,\xi)))^2)&=&   \esp(scW(i,\xi)^2) -\esp(scW(i,\xi))^2 \\
                 &=&   \int scW(i,\xi)^2 D(\xi)d\xi  -\esp(scW(i,\xi))^2      \\
                 &=&   \sum_{m=1}^{M+1} \omega_m \ scW(i,\nu_m)^2  - \left(\sum_{m=1}^{M+1} \omega_m scW(i, \nu_m)\right)^2
\eeas
\item Both exact from quadrature polynomial exactness. No simple expression for   $scW(i,\nu_m)$
\endit
}
%
\end{frame} 
%
%***************************************************************************************************

\begin{frame}{Stochastic collocation (SC) method -- 5}{Mean and variance evaluation 3/3 [skp'd]} 

\scriptsize{
%
\begit
\item Stochastic post-processing (mean and variance) done for the expansion $scW$ instead of $W$ -- In case the interpolation points $(\xi_1, \xi_2,\dots\xi_{M+1})$ are not the $M+1$ points of the $(M+1)$-point Gauss quadrature associated to $D(\xi)$
\vo
\item Interpolating quadrature associated to the set is used (it is NOT associated to distribution $D$ and $D$ terms will remain).
 Weights are denoted $(\gamma_1, \gamma_2,\dots\gamma_{M+1})$ 
\vr
\item In general, inexact evaluation of mean value (due to $D$ factor)
     $$ \esp(scW(i,\xi)) = \int scW(i,\xi) D(\xi) \xi \simeq \sum_{m=1}^{M+1}\gamma_m scW(i,\xi_m) D(\xi_m) = \sum_{m=1}^{M+1} \gamma_m W(i,\xi_m) D(\xi_m) $$ 
%
\item In general, inexact evaluation of variance (due to $D$ factor and polynomial degree)
  \beas
         \esp((scW(i,\xi)-\esp(scW(i,\xi)))^2) &=& \esp(scW(i,\xi)^2) -\esp(scW(i,\xi))^2 \\
                                     &=& \int scW(i,\xi)^2 D(\xi) d\xi -\esp(scW(i,\xi))^2 \\
                                     &\simeq& \sum_{m=1}^{M+1}\gamma_m W(i,\xi_m)^2 D(\xi_m) - \left( \sum_{m=1}^{M+1} \gamma_m W(i,\xi_m) D(\xi_m) \right)^2 \\
  \eeas
%
\endit
}
%
\end{frame} 
%
%**************************************************************************************************************
%
\begin{frame}{2D tensorial extension of SC method -- 1}{ Definition 1/2}
%
\footnotesize{
%
\begit
%
\item 2 uncertain parameters  $(\xi_1,\xi_2) \in$ \Ir$^1 \times$ \Ir$^2 $ 
%
                   $$ D(\xi_1,\xi_2) = D^{\alpha}(\xi_1)D^{\beta}(\xi_2)$$ 
%
\item For the sake of simplicity presented for a scalar output 
\item For the sake of simplicity, tensorial grid of $(M^1+1)$ and $(M^2+1)$ Gauss-points 
 associated to  $D^{\alpha}$ and $D^{\beta}$.
 $$  (\xi^\alpha_1, \xi^\alpha_2,...,\xi^\alpha_{M^1+1}) \otimes (\xi^\beta_1, \xi^\beta_2,...,\xi^\beta_{M^2+1})  $$
 the weights being 
%
 $$    (\omega^\alpha_1, \omega^\alpha_2,...,\omega^\alpha_{M^1+1})\otimes(\omega^\beta_1, \omega^\beta_2,...,\omega^\beta_{M^2+1})$$ 
%
\vo
\item Lagrange polynomials associated to the two sets
% 
  $$   H_k^\alpha(\xi_1) = \prod_{m=1, m\neq k}^{M^1+1} \displaystyle\frac{(\xi_1-\xi^\alpha_m)}{(\xi^\alpha_k-\xi^\alpha_m)}\,,\quad
  H_l^\beta(\xi_2) = \prod_{m=1, m\neq l}^{M^2+1} \displaystyle\frac{(\xi_2-\xi^\beta_m)}{(\xi^\beta_l-\xi^\beta_m)}        $$
%
\endit
%
}
%
\end{frame}
%
%
%**************************************************************************************************************
%
\begin{frame}{2D tensorial extension of SC method -- 2}{ Definition 2/2}
%
\footnotesize{
%
\begit
%
\item 2 uncertain parameters  $(\xi_1,\xi_2) \in$ \Ir$^1 \times$ \Ir$^2 $ 
%
                   $$ D(\xi_1,\xi_2) = D^{\alpha}(\xi_1)D^{\beta}(\xi_2)$$ 
%
\item $H_k^\alpha(\xi_1)$, $H_l^\beta(\xi_2)$ Lagrange polynomials associated to the two sets of $(M^1+1)$ (resp.  $(M^2+1)$) Gauss quadrature points associated to  $D^{\alpha}$  
 (resp. $D^{\beta}$)
% 
\item Stochastic collocation 2D expansion 
  $$    F(\xi_1,\xi_2) \simeq scF(\xi_1,\xi_2)  = \sum_{k \leq M^1 ; l\leq M^2} d_{kl}  H_k^\alpha(\xi_1) H_l^\alpha(\xi_2)    $$
%
\item Identification of the coefficients $d_{kl}=F(\xi^\alpha_k,\xi^\beta_l)$
  $$  scF(\xi_1,\xi_2)  = \sum_{k \leq M^1 ; l\leq M^2}  F(\xi^\alpha_k,\xi^\beta_l) H_k^\alpha(\xi_1) H_l^\beta(\xi_2)     $$  
\endit
%
}
%
\end{frame}
%
%***************************************************************************************************************
%
\begin{frame}{2D tensorial extension of SC method -- 3}
%
\footnotesize{
%
\begit
%
\item The tensor product of the two Gaussian rules is 
%
    $$ \int  F(\xi_1,\xi_2) D^\alpha(\xi_1) D^\beta(\xi_2) d\xi_1 d\xi_2 = \sum_{k \leq M^1+1 ; l\leq M^2+1} \omega^\alpha_k \omega^\beta_l F(\xi^\alpha_k,\xi^\beta_l) $$
%
\item It exactly integrates all monomials $ \xi_ 1^p   \xi_ 2^q $ such that  $p \leq 2M^1 +1$ and $q \leq 2M^2 +1$  
% 
\item Calculation of the mean of $scF$ 
 %
   $$  \int  scF(\xi_1,\xi_2) D^\alpha(\xi_1) D^\beta(\xi_2) d\xi_1 d\xi_2 =  \sum_{k \leq M^1+1 ; l\leq M^2+1} \omega^\alpha_k \omega^\beta_l scF(\xi^\alpha_k,\xi^\beta_l)   $$
%
 but simply    $scF(\xi^\alpha_k,\xi^\beta_l) = F(\xi^\alpha_k,\xi^\beta_l)$ and
%
   $$  \esp(scF) = \int  scF(\xi_1,\xi_2) D^\alpha(\xi_1) D^\beta(\xi_2) d\xi_1 d\xi_2 =  \sum_{k \leq M^1+1 ; l\leq M^2+1} \omega^\alpha_k \omega^\beta_l F(\xi^\alpha_k,\xi^\beta_l)   $$
%
\endit
%
}
%
\end{frame}
%
%***************************************************************************************************************
%
\begin{frame}{2D tensorial extension of SC method -- 4}
%
\footnotesize{
%
\begit
%
\item The tensor product of the two Gaussian quadratures  
%
    $$ \int  F(\xi_1,\xi_2) D^\alpha(\xi_1) D^\beta(\xi_2) d\xi_1 d\xi_2 = \sum_{k \leq M^1+1 ; l\leq M^2+1} \omega^\alpha_k \omega^\beta_l F(\xi^\alpha_k,\xi^\beta_l) $$
%
\item Calculation of the mean of $scF$ (exact due to polynomial degree) 
 %
   $$  \esp(scF) = \int  scF(\xi_1,\xi_2) D^\alpha(\xi_1) D^\beta(\xi_2) d\xi_1 d\xi_2 =  \sum_{k \leq M^1+1 ; l\leq M^2+1} \omega^\alpha_k \omega^\beta_l F(\xi^\alpha_k,\xi^\beta_l)   $$
%
\item Calculation of the variance  $scF$ (exact due to polynomial degree)
%
 \beas
  \Var(scF) &=&  \esp((scF-\esp(scF))^2) = \esp(scF^2)-\esp(scF)^2 \\ 
         &=&  \int  scF(\xi_1,\xi_2)^2 D^\alpha(\xi_1) D^\beta(\xi_2) d\xi_1 d\xi_2 - \esp(scF)^2  \\
        &=&   \sum_{k \leq M^1+1 ; l\leq M^2+1} \omega^\alpha_k \omega^\beta_l F(\xi^\alpha_k,\xi^\beta_l)^2  -
                                       \left( \sum_{k \leq M^1+1 ; l\leq M^2+1} \omega^\alpha_k \omega^\beta_l F(\xi^\alpha_k,\xi^\beta_l)\right)^2 \\
  \eeas
%
\endit
%
}
%
\end{frame}
%
%***************************************************************************************************************
%
\begin{frame}{$d$D tensorial generalized polynomial chaos (gPC) and stochastic collocation (SC) methods }
%
\footnotesize{
%
\begit
%
\item Assume same number of collocation or (Gaussian) quadrature points in all directions $M$
\vt
\item Calculation of polynomial expansion in dimension $d$ requires  $M^d$ CFD calculations
\vt
\item Not sustainable if $d$ is high.  Example with 9 points per direction. Required number of simulations
%
\begin{displaymath}
\begin{split}
 9^2 &= 81\,,\;9^4 = 6561\,,\;9^5=59,049\,,\;9^{6}= 531,441\,, \\
 9^{8} &= 43,046,721\,,\;9^{10}= 3,486,784,401
\end{split}
\end{displaymath}
feasible up to $d=4$ or $5$
%
\item Introduction of {\bf polynomial limited by total degree} $t$ (straightforward)
\item[] Bound the total degree $t$ of the polynomial instead of limiting the individual degree in each variable. Number of 
terms of the basis 
 $$ Z = \binom{d+t}{t} $$
\vo
\item Introduction of Smolyak's {\bf sparse quadratures}\footnote{\href{http://mi.mathnet.ru/eng/dan/v148/i5/p1042}{\scriptsize{S. A. Smolyak: Quadrature and interpolation formulas for tensor products of certain classes of functions. {\sl Soviet Math. Dokl.} {\bf 4}, 240-243 (1963)}}} often called sparse grids (not so simple)
%
\endit
%
}
%
\end{frame}
%
%
%%%%%%%%%%%%%%%%%%%%%%%%%%%%%%%%%%%%%%%%%%%%%%%%%%%%%%%%%%%%%%%%%%%%%%%%%%%%%%%%%%%%%%
%
\section{Introduction to Smolyak's sparse quadratures}
%
%%%%%%%%%%%%%%%%%%%%%%%%%%%%%%%%%%%%%%%%%%%%%%%%%%%%%%%%%%%%%%%%%%%%%%%%%%%%%%%%%%%%%%
%
%
\begin{frame}{Smolyak sparse grids -- 1}{ Tensorial product of two quadrature rules (Reminder)}
%
\footnotesize{
%
\begit
%
\item 2D case 
% 
 $$ A[f]=\sum_{i=1}^m a_i f(x_i)\,,\quad B[f] =\sum_{i=1}^n b_i f(y_i),  $$ 
%
$$ A \otimes B[g] = \sum_{i=1}^m \sum_{j=1}^n a_i b_j~ g(x_i,y_j)\,. $$
%
\item Straightforward extension to $d$D 
%
%
$$  A_{1}\otimes A_2 \otimes\cdots\otimes A_{d}[f] = \sum_{i_1=1}^{n_1}\sum_{i_2=1}^{n_2}\cdots\sum_{i_d=1}^{n_{d}} w_{1i_1}w_{2i_2}\dots w_{di_d}\,f(x_{1i_1},x_{2i_2},\dots x_{di_d}) $$
%
\endit
%
}
%
\end{frame}
%
%***************************************************************************************
%
\begin{frame}{Smolyak sparse grids -- 2}{Hierarchy of 1D quadratures. Difference of 1D quadratures}
%
\footnotesize{
%
\begit
%
\item 1D hierachy of quadratures denotes $Q_l$ with increasing number of points. Assumed to be used in all directions
%
\item  Nested (= quadrature points of points $Q_l$ include the quadrature points of $Q_ {l-1}$) or not nested
% 
\vt
\item Difference in successive quadratures
\begin{displaymath}
\begin{split}
   \Delta_k [f] &:= Q_k [f] - Q_{k-1} [f] \,,\\
        Q_0 [f]  &:= 0\,.
\end{split}
\end{displaymath}
%
\item {\bf Rewriting of a tensor quadrature} (since $\smash{\sum_{k=1}^n\Delta_k[f]=Q_n[f]}$)
 $$  Q_{l_1}\otimes\cdots\otimes Q_{l_d}[f] = \sum_{{\bf k}/~ 1\leq k_j \leq l_j} \Delta_{k_1} \otimes\cdots\otimes \Delta_{k_d}[f] $$
%
\item[] 2D illustration
\begin{displaymath}
\begin{split}
  Q_3\otimes Q_2 [f] = &  \quad\;(Q_3-Q_2) \otimes (Q_2-Q_1) [f] + (Q_3-Q_2) \otimes (Q_1-Q_0) [f]  \\
                      & + (Q_2-Q_1) \otimes (Q_2-Q_1) [f] + (Q_2-Q_1) \otimes (Q_1-Q_0) [f]   \\
                      & + (Q_1-Q_0) \otimes (Q_2-Q_1) [f] + (Q_1-Q_0) \otimes (Q_1-Q_0) [f] 
\end{split}
\end{displaymath}
\endit
%
}
%
\end{frame}
%
%***************************************************************************************
%
\begin{frame}{Smolyak sparse grids -- 3}{Smolyak sparse quadratures (1/2)}
%
\footnotesize{
%
\begit
%
\item Fundamental rewriting of a tensor quadrature
 $$  Q_{l_1}\otimes\cdots \otimes Q_{l_d}[f] = \sum_{{\bf k}/~ 1\leq k_j \leq l_j} \Delta_{k_1} \otimes \cdots \otimes \Delta_{k_d}[f]  $$
%
\vt
\item Definition of {\bf Smolyak's sparse quadrature of level  $l$ }
$$  Q_l^d [f] = \sum_{|{\bf k}|_1 \leq l+d-1} \Delta_{k_1} \otimes \cdots \otimes \Delta_{k_d} [f] $$ 

\vt
\item Very general construction referring to the {\bf indices of the 1D quadratures in the hierarchy (not degree, not polynomial exactness...) }
\endit
%
}
%
\end{frame}
%
%***************************************************************************************
%
\begin{frame}{Smolyak sparse grids -- 4}{Smolyak's sparse quadratures (2/2)}
%
\footnotesize{
%
\begit
%
\item Definition of {\bf Smolyak's sparse quadrature of level  $l$ }
$$  Q_l^d [f] = \sum_{|{\bf k}|_1 \leq l+d-1} \Delta_{k_1} \otimes ...  \otimes \Delta_{k_d} [f] $$ 
\vt
\item Other expressions of Smolyak's sparse grids with difference of quadratures
\begin{displaymath}
\begin{split}
Q_l^d [f]  &= \sum_{j=d}^{d+l-1} \sum_{{\bf k}/|{\bf k}|_1=j} \Delta_{k_1} \otimes \cdots \otimes \Delta_{k_d} [f] \\
Q_{l+1}^d [f]  &=   Q_l^d [f]+  \sum_{{\bf k}/|{\bf k}|_1=d+l} \Delta_{k_1} \otimes \cdots \otimes \Delta_{k_d} [f] 
\end{split}
\end{displaymath}

\item Direct expressions of Smolyak's sparse grids with quadratures\footnote{\href{\webDOI/10.1006/jcom.1995.1001}{\scriptsize{G. W. Wasilkowski, H. Wo\'zniakowski: Explicit cost bounds of algorithms for multivariate tensor product problems. {\sl J. Complexity} {\bf 11}(1), 1-56 (1995)}}}
$$
   Q_l^d [f] = \sum_{\max(l,d)\leq |{\bf k}|_1 \leq l+d-1} (-1)^{l+d-|{\bf k}|_1-1}\binom{d-1}{|{\bf k}|_1-l}
%   \left (
%  \begin{tabular}{c}
%   $ d-1$ \\
%   ${|{\bf k}|}_1-l $
%  \end{tabular}
%  \right)
  Q_{k_1} \otimes \cdots \otimes Q_{k_d} [f]
$$
\endit
%
}
%
\end{frame}
%
%***************************************************************************************
%
\begin{frame}{Smolyak sparse grids -- 5}{Polynomial exactness (1/2)}
%
\footnotesize{
%
\begit
%
\item  Tensorial product of 1D polynomials 
$$ \bigotimes_{i=1}^d \mathbb{P}^1_{s_i}=\ds\left\{\Rset^d\ni(x_1, ..., x_d) \mapsto \prod_{i=1}^d p_i(x_i)\in \Rset\,,\;p_i \in {\cal P}_{s_i}^1 \right\}$$
%
where ${\cal P}_{s_i}^1$ is the set of mono-variable polynomials of degree lower or equal to $s_i$  
\vt
\item The $i$-th quadrature of the 1D hierarchy $\smash{Q_ i}$ is assumed to have polynomial exactness $\smash{m_i}$ such that $\smash{m_i} \leq \smash{m_{i+1}}$
\vt
\item Smolyak's sparse grid quadrature
%
$$  Q_l^d [f] = \sum_{|{\bf k}|_1 \leq l+d-1} \Delta_{k_1} \otimes\cdots \otimes \Delta_{k_d} [f] $$
%
 is exact for all polynomials of the non classical space
%
    $$  {\cal V}(Q_l ^d)  = \text{Span} \{ {\cal P}^1_{m_{k_1}} \otimes\cdots\otimes {\cal P}^1_{m_{k_d}}\,;\; |{\bf k}|_1 = l+d-1 \} $$
\endit
%
}
%
\end{frame}
%%
%
%***************************************************************************************
%
\begin{frame}{Smolyak sparse grids -- 6}{Polynomial exactness (2/2)}
%
\footnotesize{
%
\begit
%
\item  Example: Series of nested ($n$ to $n+2$) rules $U_1$, $U_2$, $U_3$, $U_4$ involving $n_1=1$, $n_2=3$, $n_3=5$, $n_4=7$ points
 and having polynomial exactness $m_1=0$, $m_2=2$, $m_3=4$, $m_4=6$
%
\vf
\item Polynomial exactness of Smolyak's sparse grid  $U_4^2$
\begin{displaymath}
\begin{split}
  U_4^2 [f] &= \sum_{j=2}^{5}\;\sum_{{\bf k}/|{\bf k}|_1=j} \Delta_{k_1} \otimes \Delta_{k_2} [f] \\
  &= ( U_4\otimes U_1 + U_3 \otimes U_2 + U_2 \otimes U_3 + U_1\otimes U_4 + \text{lower order}) [f] 
\end{split} 
\end{displaymath}
%
\vf
\item From previous slide, $U_4^2$ is exact for polynomial vector space  ${\cal V }(U_4 ^2)$
%
 $$  {\cal V }(U_4 ^2) =\text{Span} \{ {\cal P}_{m_4}\otimes {\cal P}_{m_1} + {\cal P}_{m_3} \otimes {\cal P}_{m_2} +
                            {\cal P}_{m_2} \otimes {\cal P}_{m_3} + {\cal P}_{m_1}\otimes {\cal P}_{m_4}   \}\,, $$
%
 that is
%
 $$  {\cal V }(U_4 ^2) = \text{Span} \{ {\cal P}_{6}\otimes {\cal P}_{0} + {\cal P}_{4}\otimes {\cal P}_{2} +
                            {\cal P}_{2} \otimes {\cal P}_{4} + {\cal P}_{0}\otimes {\cal P}_{6}   \}\,. $$
\endit
%
}
%
\end{frame}
%%
%***************************************************************************************
%
\begin{frame}{Smolyak sparse grids -- 7}{Number of evaluations, bounds for weights, error analysis }
%
\footnotesize{
%
\begit
%
\item  Number of evaluations, bounds for weights, error analysis require analysis for each individual family of quadrature $\smash{Q_ l}$ 
%
\vt
\item Classical results for Clenshaw-Curtis
  \begin{displaymath}
\left\{ \begin{split}
 n_l &= 2^{l-1}+1\,,\quad l>1\,, \\
 n_1 &= 1\,,
\end{split}\right.
\end{displaymath}
and $m_l=n_l-1$.
\vt
\item For fixed dimension $d$ and $l \rightarrow \infty $ the number of points involved in $\smash{Q^d_l}$, denoted by $n(Q^d_l)$, is equivalent (strong sense of limit
 of sequences being equal to 1) to: 
%
 $$  n(Q^d_ l) \simeq  \frac{1}{(d-1)! \,2^{d-1}} 2 ^{l-1} (l-1)^{d-1} $$
%
\item[] Maximum number of points along all axis (obtained for one $\smash{k_j}$ equal $l$ all the other equal 1) equal $\smash{2^{l-1}+1 }$.
  ``Corresponding'' tensorial number of points $ (2^{l-1}+1)^d $
\vt
\item Error estimation depending on function regularity. See Novak-Ritter 1997 (possibly Dumont-Le Brazidec-Peter 2018)
\endit
%
}
%
\end{frame}
%
%
%%%%%%%%%%%%%%%%%%%%%%%%%%%%%%%%%%%%%%%%%%%%%%%%%%%%%%%%%%%%%%%%%%%%%%%%%%%%%%%%%%%%%%
%
\section{Examples of application}
%
%%%%%%%%%%%%%%%%%%%%%%%%%%%%%%%%%%%%%%%%%%%%%%%%%%%%%%%%%%%%%%%%%%%%%%%%%%%%%%%%%%%%%%
%
\begin{frame}{FG5 generic missile -- 3 uncertain parameters}{Nominal mesh at the wall }  
%
\begit
\item Generic missile FG5; joint \Onera, DLR, USAF exercise\footnote{\href{\webDOI/10.2514/6.2017-1197}{\scriptsize J. Peter, S. G\"ortz, R. E. Graves: Three-parameter uncertainty quantification for generic missile FG5. AIAA Paper \#2017-1197 (2017)}}
\item \Onera\ experiments.  RANS CFD
\item 3 uncertain parameters exercise: angle of attack $\alpha$, upper fin angle, upper fin position
\item Three outputs of interest: side force (CYA), rolling moment (CLA), yawing moment (CNA)
\endit
%
\begin{figure}[!h]
\begin{center}
\includegraphics[width=4.5cm]{\FIGS/Figures-FG5/missile.jpg}
\end{center}
\end{figure}
%
\end{frame} 

%================================================================================

\begin{frame}{FG5 generic missile -- 3 uncertain parameters}{Flow conditions. Uncertain parameters}  

\begit
\item Nominal flow conditions: $\smash{\Mach_\infty}=0.8$, $\alpha=\smash{12^\circ}$, $Re_D=0.6~ 10^6$ 
\item Output of interest: rolling moment, yawing moment, side force
\item Uncertain parameters
  \begit
  \item Angle of attack in $[10^\circ,14^\circ]$ with PDF:
   $$  d\alpha' = (\alpha-12)/2\,,\quad D^{s2}(d\alpha')= \frac{15}{16}(1-d\alpha'^2)^2\,; $$
  \item Change in upper fin azimutal position in $[-1^\circ,1^\circ]$ with PDF: 
    $$  d\phi = \phi- 22.5\,,\quad D^{s3}(d\phi)= \frac{35}{32}(1-d\phi^2)^3\,; $$
  \item Upper fin angle in $[-1^\circ,1^\circ]$ with PDF:
    $$  D^{s3}(\xi)= \frac{35}{32}(1-\xi^2)^3\,. $$ 
  \endit
\item Joint probability of the three uncertain parameters:
  $$\!\!\!\!\!\!\!\! D(d\alpha',d\phi,\xi)= D^{s2}(d\alpha')D^{s3}(d\phi)D^{s3}(\xi)=\frac{15}{16}\frac{35^2}{32^2} (1-d\alpha'^2)^2 (1-d\phi^2)^3 (1-\xi^2)^3  $$ 
\endit
%
\end{frame} 
%
%================================================================================
%
\begin{frame}{FG5 generic missile -- 3 uncertain parameters}{Outputs of interest as function AoA}  
%
\begin{figure}[!h]
\begin{center}
\includegraphics[width=4.9cm]{\FIGS/Figures-FG5/SideForce.png}
\includegraphics[width=5.cm]{\FIGS/Figures-FG5/RollingMoment.png}
\\
\includegraphics[width=5.cm]{\FIGS/Figures-FG5/YawingMoment.png}
\end{center}
\end{figure}
%
\end{frame} 
%
%================================================================================
%
%
\begin{frame}{FG5 generic missile -- 3 uncertain parameters}{Nominal flow (1/2)}  
%
\begin{figure}[!h]
\begin{center}
\includegraphics[width=4.cm]{\FIGS/Figures-FG5/pi_piinf_x_0_64.png}
\includegraphics[width=4.cm]{\FIGS/Figures-FG5/pi_piinf_x_1_173333.png}
\\
\includegraphics[width=4.cm]{\FIGS/Figures-FG5/pi_piinf_x_1_226666.png}
\includegraphics[width=4.cm]{\FIGS/Figures-FG5/pi_piinf_x_1_27999.png}
\end{center}
\end{figure}
%
\end{frame} 
%
%================================================================================
%
\begin{frame}{FG5 generic missile -- 3 uncertain parameters} {Nominal flow (2/2)}
% 
\begin{figure}[!h]
\begin{center}
\includegraphics[width=4.cm]{\FIGS/Figures-FG5/KP_cylinder_fins_phi_m_112_5.jpg}
\includegraphics[width=4.cm]{\FIGS/Figures-FG5/KP_cylinder_fins_phi_m_22_5.jpg}
\\
\includegraphics[width=4.cm]{\FIGS/Figures-FG5/KP_cylinder_fins_phi_p_67_5.jpg}
\includegraphics[width=4.cm]{\FIGS/Figures-FG5/KP_cylinder_fins_phi_p_157_5.jpg}
\end{center}
\end{figure}
%
\end{frame} 
%
%================================================================================

\begin{frame}{FG5 generic missile -- 3 uncertain parameters}  

\begit
\item Comparison of nominal flows
    \begit
    \item DLR Calculations with TAU, USAF calculations with AVUS, \Onera\ calculations with \elsA\
    \item Comparison of pressure coefficients $K_p$ on the fins and rear part of the missile, comparison of stagnation pressure in vertical planes
    \item The three flow solutions match well. Good starting point for UQ study
    \endit
\vs  
\item Individual variation of outputs w.r.t. parameters
   \begit
   \item CYA, CLA, CNA non linear as function of $\alpha$ as in the experiment 
   \item CYA, CLA, CNA linear as function of fin angle
   \item CYA, CLA, CNA linear as function of fin position
   \endit
\endit
%
\end{frame} 
%

%================================================================================
%
\begin{frame}{FG5 generic missile -- 3 uncertain parameters} {Fin deformation}
%
 (the two mesh deformations can be combined)
%
\begin{figure}[!h]
\begin{center}
\includegraphics[width=5.2cm]{\FIGS/Figures-FG5/meshdef_finangle_1.png}
\hspace{3mm}
\includegraphics[width=5.2cm]{\FIGS/Figures-FG5/meshdef_finposition.png}
\end{center}
\end{figure}
%
\end{frame}
%
%================================================================================

\begin{frame}{FG5 generic missile -- 3 uncertain parameters}{Strategies for UQ}  

\begit
%
\item ONERA
  \vt 
  \begit
  \item 3D quadrature = 31-point Smolyak sparse grid  based on (1D) F\'ejer second rule.
  \item 31 flow calculations. Classical checks  
  \vr
  \item Quadrature exactly integrates degree 3 polynomials in dimension 3... but $D(d\alpha',d\phi,\xi)$
     is a degree 16 polynomial 
  \item Considered quadrature fails to correctly integrate $D(d\alpha',d\phi,\xi)$
  \vr
  \item Kriging fitted to the 31 evaluations of CLA. Corresponding surrogates for CYA, CNA
  \item Calculation of mean value and variance based on Riemann sums for (surrogate $\times $  $D(d\alpha',d\phi,\xi)$)  
  \endit
%
\endit
%
\end{frame} 
%
%================================================================================

\begin{frame}{FG5 generic missile -- 3 uncertain parameters}{Strategies for UQ}  

\begit
%
\item DLR
  \begit
  \item 76 TAU simulations budget (actually 8, 16, 32,88, 64 then 76 performing intermediate statistics)
  \item Three Kriging surrogates fitted to the 8 then 16... then 76 CLA, CYA, CNA values
  \item One million Monte-Carlo sample built from the cumulative density functions of  $D^{s2}(d\alpha')$, $D^{s3}(d\phi)$ and $D^{s3}(\xi)$
  \item Monte-Carlo mean and variance for the Kriging surrogates based on the  $D(d\alpha',d\phi,\xi)$-consistent sampling
  \item (Visual) PDF of outputs
  \vt
  \endit
\vv
\item ONERA - DLR
  \begit
  \item Checking individual variations of CLA, CYA, CNA w.r.t. ONERA calculations showed differences in slopes 
 $\rightarrow$ differences in variance expected.
  \endit
%
\endit
%
\end{frame} 
%
%
%================================================================================

\begin{frame}{FG5 generic missile -- 3 uncertain parameters}{Strategies for UQ}  

\begit
%
\item USAF
  \begit
  \item 10 simulations budget 
  \item DoE = corners of the parameters domain plus two  face centers
  \item Quadratic surrogate
  \item Analysis of variance based on the quadratic surrogate
  \endit
%
\endit
%
\end{frame} 
%
%================================================================================

\begin{frame}{FG5 generic missile -- 3 uncertain parameters}{}  

More difference in standard deviation than in mean (than visually looking at $Kp$)   
%
\begin{figure}[!h]
\begin{center}
\includegraphics[width=11.7cm]{\FIGS/Figures-FG5/synthese-stoch-ONERA.png}
\vr\\
\includegraphics[width=11.7cm]{\FIGS/Figures-FG5/synthese-stoch-DLR.png}
\end{center}
\end{figure}
%
\end{frame} 
%
%============================================================================================
%
\begin{frame}{RAE2822 -- 3 uncertain parameters}{Nominal mesh at the wall}  
%
\footnotesize{
\begit
\item RAE2822,\footnote{\href{\webDOI/10.2514/6.2016-0433}{\scriptsize{\'E.~Savin, A.~Resmini, J.~Peter: Sparse polynomial surrogates for aerodynamic computations with random inputs. AIAA Paper \#2016-0433 (2016)}}} UMRIDA EU project (\href{http://www.umrida.eu}{www.umrida.eu})
\item RAE experiments.  
\item Case 6. Flow conditions  $\nominal{\Mach}_\infty=0.725$,  $\nominal{\alpha}=2.92^\circ$, $Re=6.50\cdot10^6 $
\item RANS calculations. Outputs of interest $\drag$, $\lift$, $\moment$
\item Uncertainties on free-stream Mach number $\nominal{\Mach}_\infty$, angle of attack $\nominal{\alpha}$, thickness to chord ratio $\nominal{r}=h/c$
\endit
%
\begin{table}[h!]
\begin{center}
\begin{tabular}{|c||c|c|c|c|}
\hline
& \makebox[3em]{$a=b$} &  \makebox[3em]{$X_m$} & \makebox[3em]{$X_M$} \\
\hline\hline
$\xigj_1$ & $4$ & $0.97\times\nominal{r}$ & $1.03\times \nominal{r}$ \\
$\xigj_2$ & $4$ & $0.95\times\nominal{\Mach}_\infty$ & $1.05\times\nominal{\Mach}_\infty$ \\
$\xigj_3$ & $4$ & $0.98\times\nominal{\alpha}$ & $1.02\times\nominal{\alpha}$ \\
\hline
\end{tabular}
\end{center}
\end{table}
%
$$ \PDFb(x;a,b)={\mathbbm 1}_{[X_m,X_M]}(x)\frac{\Gamma(a+b)}{\Gamma(a)\Gamma(b)}\frac{(x-X_m)^{a-1}(X_M-x)^{b-1}}{(X_M-X_m)^{a+b-1}} $$
%
}
%
\end{frame} 

%================================================================================
%
\begin{frame}{RAE2822 -- 3 uncertain parameters} {Mesh}
%
\begin{figure}[!h]
\begin{center}
\includegraphics[width=5.6cm]{\FIGS/Figures-RAE/GridNom.png}
\hspace{3mm}
\includegraphics[width=5.6cm]{\FIGS/Figures-RAE/GridNomZoom.png}
\end{center}
\end{figure}
%
\end{frame}
%
%================================================================================
%
\begin{frame}{RAE2822 -- 3 uncertain parameters} {Mesh}
%
\begin{figure}[!h]
\begin{center}
\centering
\includegraphics[width=5.6cm]{\FIGS/Figures-RAE/contour_M_nominal}
\hspace{3mm}
\includegraphics[width=5.6cm]{\FIGS/Figures-RAE/Cpexp-num}
\end{center}
\end{figure}
%
\end{frame}
%
%
%============================================================================================
%
\begin{frame}{RAE2822 -- 3 uncertain parameters}{gPC expansion of outputs of interest [typo]}  
%
\footnotesize{
\begit
%
\item gPC expansion. Normalized 1D Jacobi-polynomials $\psi$ orthonormal for   
                  $$<\psi_j,\psi_k>=\int_{-1}^{+1}\psi_j(\xi)\psi_k(\xi)~ \frac{35}{32}(1-\xi^2)^3 d\xi = \delta_{jk}$$
%
\item Multivariate polynomials involved in the $\Rset^3 \to \Rset^1$ expansions of $\drag$, $\lift$, $\moment$
 $$ \psi_{\bf j}({\bf \xi})=\prod_{d=1}^3\psi_{j_d}(\xi_d)\,, \quad | {\bf j} |_1 = j_1 + j_2 + j_3 \leq t $$
%
\item Total degree is $\torder=8$. Dimension is $\dimrv=3$. Number of term in the polynomial expansion is
    $$ Z = \binom{\torder+\dimrv}{\dimrv} = \binom{8+3}{3} = \binom{11}{3} = 165  $$
\vo
\item $\drag$ gPC expansion
 $$ g\drag(\xi_1, \xi_2, \xi_3) = \sum_{| {\bf j} |_1=j_1+j_2+j_3 \leq 8} c_{\bf j}~ \psi_{j_1}(\xi_1) ~ \psi_{j_2}(\xi_2) ~ \psi_{j_3}(\xi_3)  $$ 
\endit
}
%
\end{frame} 
%
%============================================================================================
%
\begin{frame}{RAE2822 -- 3 uncertain parameters}{gPC expansion of outputs of interest }  
%
\begit
%
\item 1D base-quadrature = $p$-point Gauss-Jacobi-Lobatto (GJL) quadrature. Polynomial exactness degree ($2p-3$)
%
\vv
%
%
\item 3D quadratures
\vt
   \begit
   \item Tensorial grid = Tensorial product of the $10$-point GJL quadrature. 
   \item[] Polynomial exactness up to degree $17$ for each variable $\smash{\xi_j}$. Exact integration of products of degree 8 polynomials. 
    Exact variance of gPC expansions.
   \item[] Number of points $1000$ $(=10^3)$
    \vt
%
    \item Smolyak sparse grid = 7-th level Smolyak sparse grid based on the family of GJL quadratures.
   \item[] Number of points $201$
    \endit
%
\endit
%
\end{frame} 
%
%================================================================================
%
\begin{frame}{RAE2822 -- 3 uncertain parameters} {Visualization of quadrature points}
%
\begin{figure}[!h]
\begin{center}
\includegraphics[width=12cm]{\FIGS/Figures-RAE/stencil-2-quadratures.png}
\caption{Visualization of $6$-point tensorial GJL quadrature and $7$-th level Smolyak quadrature based on  GJL nodes--gPC coefficients are calculated with $10$-point tensorial GJL quadrature and $7$-th level Smolyak quadrature based on GJL nodes}
\end{center}
\end{figure}
%
\end{frame}
%
%============================================================================================
%
\begin{frame}{RAE2822 -- 3 uncertain parameters}{gPC expansion of outputs of interest }  
%
\scriptsize{
\begit
%
\item Calculation of the gPC coefficients as (for $\drag$) 
\beas
  c_{\bf j}& =& \int  \psi_{{\bf j}}({\bf \xi}) \drag({\bf \xi}) D({\bf \xi})d{\bf \xi} \\
           &=& \int  \psi_{j_1}(\xi_1) ~ \psi_{j_2}(\xi_2) ~ \psi_{j_3}(\xi_3) \drag(\xi_1,\xi_2,\xi_3) \frac{35^3}{32^3}(1-\xi_1^2)^3 (1-\xi_2^2)^3 (1-\xi_3^2)^3 
  d\xi_1 d\xi_2 d\xi_3 
\eeas
%    
\vt
\item First two moments of the aerodynamic coefficients computed by the $10$--th level product rule (1000 points)
\begin{table}[h!]
\begin{center}
\begin{tabular}{|c||c|c|}
\hline
& \makebox[3em]{$\mu$} &  \makebox[3em]{$\sigma$}  \\
\hline\hline
$\drag$ & 133.37e-04 & 34.128e-04  \\
$\lift$ & 72.274e-02 & 1.6695e-02  \\
$\moment$ & -453.99e-04 &  32.239e-04 \\
\hline
\end{tabular}
\end{center}
\end{table}
%
\item First two moments of the aerodynamic coefficients computed by the $7$--th level sparse rule  (201 points)
\begin{table}[h!]
\begin{center}
\begin{tabular}{|c||c|c|}
\hline
 & \makebox[3em]{$\mu$} &  \makebox[3em]{$\sigma$} \\
\hline\hline
$\drag$ & 133.38e-04 & 34.097e-04  \\
$\lift$ &  72.269e-02 & 1.6729e-02  \\
$\moment$ & -453.96e-04 & 32.175e-04 \\
\hline
\end{tabular}
\end{center}
\end{table}
%
\endit
}
%
\end{frame} 
%
%%*******************************************************************************************************

\begin{frame}{RAE2822 -- 3 uncertain parameters}{gPC compressed sensing (1/4)}  
%
\footnotesize{
%
\begit

\item Reminder: calculation of gPC coefficients by collocation. 
\item Presentation in case of a multi-variate polynomial of fixed total order
 \vf
\item[] Identify $F({\bf \xi})$ and  $gF({\bf \xi})$ for $q$ values of ${\bf \xi}$. 

  $$  \sum_{ | {\bf j} |_1 \leq t} C_{\bf j} P_{\bf j}(\xig_k) = F(\xig_ k)\,,\quad\forall k \in \llbracket 1,q\rrbracket $$

\vt
\item Matrix notation $ {\bf F}  $ column vector of $F$ values, ${\bf C}$ column vector of unknown polynomial coefficients 
 ${\bf K}$ matrix $K_{i{\bf j}}=  P_{\bf j}(\xi_i) $
%
    $$   {\bf K}{\bf C} = {\bf F}  $$
\item Square linear system if number of evaluations $=$ dimension polynomial basis 
\item Least square system if number of evaluations $>$ dimension polynomial basis
\item {\bf Possible use of  compressed sensing\footnote{\href{\webDOI/10.1109/MSP.2007.914731}{\scriptsize{E. J. Cand\`es, M. B. Wakin: An introduction to compressive sampling. {\sl IEEE Sig. Proc. Mag.} {\bf 25}(2), 21-30 (2008)}}} if number of evaluations $<$ dimension polynomial basis} 
%
\endit
%
}
%
\end{frame} 
%
%%*******************************************************************************************************
%
\begin{frame}{RAE2822 -- 3 uncertain parameters}{gPC compressed sensing (2/4)}  
%
\footnotesize{
%
\begit
\item[] Collocation linear system. Identify $F({\bf \xi})$ and  $gF({\bf \xi})$ for $q$ values of ${\bf \xi}$. 
%
  $$ F(\xig_k)= \sum_{ | {\bf j} |_1 \leq t} C_{\bf j} P_{\bf j}(\xig_k)\,,\quad\forall k \in \llbracket 1,q\rrbracket $$
%
or in matrix notation %
    $$   {\bf K}{\bf C} = {\bf F}  $$
${\bf K}$ has $q$ lines (number of evaluations) and  $Z$ columns (number of polynomials in the basis)  
\vt
\item May be solved with less information (evaluations) than unknowns (gPC coefficients) by {\bf compressed sensing} provided
   \begit
   \item The actual $gPC$ expansion that is looked for is sparse $=$ has many coefficients very close to 0. 
 This is often the case. This is called "{\bf sparsity of effects}." This is verified for the searched expansion  
   \item Requires a (random) {\bf sampling incoherent with basis of polynomial} that is measured by the ``mutual coherence''
   $$ \max_{\tiny\begin{array}{c}1\leq j,l\leq Z \\ j\neq l\end{array}}\frac{| K_j^T K_l|}{ \| K_j\|_2 \| K_l \|_2}\ $$
   that should have the lowest possible value
   \endit
%
\endit
%
}
%
\end{frame} 
%
%%*******************************************************************************************************
%
\begin{frame}{RAE2822 -- 3 uncertain parameters}{gPC compressed sensing (3/4)}  
%
%\footnotesize{
%
\begit
\item Collocation linear system. Identify $F({\bf \xi})$ and $gF({\bf \xi})$ for $q$ values of ${\bf \xi}$. 

  $$ F(\xig_k)= \sum_{ | {\bf j} |_1 \leq t} C_{\bf j} P_{\bf j}(\xig_k)\,,\quad\forall k \in \llbracket 1,q\rrbracket $$

\vt
\item In matrix notation %
    $$   {\bf K}{\bf C} = {\bf F}  $$
\vt
\item The underdetermined problem is then solved by $\ell_1$ minimization 
%
$$   {\bf C}^* =\arg\min_{{\boldsymbol h}\in {\mathbb R}^Z}\{\| {\bf h} \|_1;\;\| {\bf K}{\bf h} - {\bf F} \|_2 \leq \epsilon \} $$
\endit
%
%}
%
\end{frame} 
%
%%*******************************************************************************************************
%
\begin{frame}{RAE2822 -- 3 uncertain parameters}{gPC compressed sensing (4/4)}  
%
%\footnotesize{
\begit
%
\item 165 polynomials in the basis
\vo
\item 80 random sampling points
\vo
\item Mutual coherence equal 0.93 
\vo
\item Good recovery of mean and variance with compressed sensing gPC:
\vo
\begin{table}[h!]
\begin{center}
\begin{tabular}{|c||c|c|}
\hline
 & \makebox[3em]{$\mu$} &  \makebox[3em]{$\sigma$} \\
\hline\hline
$\drag$ & 133.33e-04 & 34.052e-04 \\
$\lift$ & 72.271e-02 & 1.6703e-02 \\
$\moment$ & -453.95e-04 & 32.180e-04 \\
\hline
\end{tabular}
\end{center}
\end{table}
%
\endit
%}
%
\end{frame} 
%

%\section{Application to ALBATROS flexible wing-fuselage configuration (3D)}

\begin{frame}{ALBATROS\footnote{\href{http://www.icas.org/ICAS_ARCHIVE/ICAS2012/PAPERS/597.PDF}{\scriptsize{G. Carrier, O. Atinault, S. Dequand, J.-L. Hantrais-Gervois, C. Liauzun, B. Paluch, A.-M. Rodde, C. Toussaint. Investigation of a strut-braced wing configuration for future commercial transport. ICAS Paper 2012-1.10.2 (2012)}}}
 flexible wing-fuselage configuration (3D)}{Numerical model}

\footnotesize{
\begin{itemize}
\item RANS + Spalart-Allmaras model, $9,008,512$ cells:
\begin{figure}
\centering\includegraphics[width=6cm]{\figUMRIDA/MeshAlbatros}
\end{figure}
%\vspace{0.5truecm}
\item Implicit LU-SSOR phase; Jameson centered scheme with additional artificial viscosity outside of the boundary layers; Multigrid approach for the NS system over $3$ grid levels; Backward Euler time integration scheme;
\item Typical computational time: 6 hours on 60 cores.
\end{itemize}
%\vskip5pt
%\hfill\mycite{\href{\webDOI/10.1051/meca/2013056}{Cambier-Heib-Plot \emph{Mechanics \& Industry} {\bf 14}(3), 159 (2013)}}
}
\end{frame}

\begin{frame}{ALBATROS flexible wing-fuselage configuration (3D)}{Definition of the uncertainties ($\dimrv=10$)}

\vskip-10pt
\begin{itemize}
\item The Mach number $\Mach\equiv\smash{\xigj_1}$, lift coefficient $\smash{\lift\equiv\xigj_2}$, 4 wing bending parameters $\smash{\xigj_3,\dots\xigj_6}$, and 4 wing torsion parameters $\smash{\xigj_7,\dots\xigj_{10}}$ are $\dimrv=10$ variable parameters following uniform marginal probability laws.
\end{itemize}

\begin{table}[h!]
\begin{center}
\begin{tabular}{|c||c|c|c|}
\hline
&  \makebox[3em]{$X_m$} & \makebox[3em]{$X_M$} \\
\hline\hline
$\xigj_1$ & $0.74$ & $0.76$ \\
$\xigj_2$ & $0.55$ & $0.65$ \\
$\xigj_3\dots\xigj_6$ & $0.5\times\nominal{I}$ & $2.0\times\nominal{I}$ \\
$\xigj_7\dots\xigj_{10}$ & $0.02\times\nominal{J}$ & $10.0\times\nominal{J}$ \\
\hline
\end{tabular}
\end{center}
\end{table}

\begin{itemize}
\item Our aim is to construct polynomial surrogates for the angle of attack $\AoA$, drag coefficients $\Dragj{s}$ and $\Dragj{v}$ computed at the wing skin and in the far-field, respectively, pitching moment coefficient $\Pitch$, wing tip bend $\Bend$, and wing tip twist $\Twist$ using PC adapted to the foregoing PDFs (Legendre polynomials).\footnote{\href{\webDOI/10.1016/j.probengmech.2020.103027}{\scriptsize{\'E. Savin, J.-L. Hantrais-Gervois. Sparse polynomial surrogates for non-intrusive, high-dimensional uncertainty quantification of aeroelastic computations. \emph{Prob. Engng. Mech.} {\bf 59}, 103027 (2020)}}}
\end{itemize}

\end{frame}

\begin{frame}{ALBATROS flexible wing-fuselage configuration (3D)}{Sampling set (design of experiments DoE)}

\begin{itemize}
\item The DoE is constituted by a combination of $18$ manually generated sampling points (\textcolor{red}{$\bullet$}) and $83$ randomly generated sampling points (\textcolor{red}{$\circ$}) using Latin Hypercube Sampling (LHS).

\end{itemize}
\begin{columns}
\column{.5\textwidth}
\begin{figure}
\centering{\includegraphics[scale=0.35]{\figc/Mach-CL}}
\end{figure}
\column{.5\textwidth}
\begin{figure}
\centering{\includegraphics[scale=0.35]{\figc/B4-T4}}
\end{figure}
\end{columns}

\end{frame}

\begin{frame}{ALBATROS flexible wing-fuselage configuration (3D)}{Output statistics by $\ell_1$--minimization ($\torder=285$, $\nquad=101$)}

\begin{itemize}

\item Mean, variance, root-mean square error, and Kullback-Leibler divergence from a Normal distribution $\PDFN$:
\begin{center}
{\tiny\begin{tabular}{|c||c|c|c|c|c|c|}
\hline
& \makebox[5em]{$\Pitch$} & \makebox[5em]{$\Dragj{s}$} & \makebox[5em]{$\Dragj{v}$} & \makebox[5em]{$\AoA$} & \makebox[5em]{$\Bend$} & \makebox[5em]{$\Twist$}\\
\hline\hline
$\mu$ & -11.63e-02 & 219.83e-04 & 218.69e-04 & 2.53  & 2.16 & -6.10 \\
$\sigma$ & 1.55e-02 & 6.53e-04 & 6.28e-04 &  0.20 & 0.26 & 0.80 \\
$e_2$ & 4.70e-03 &  0.45e-03 & 0.35e-03 & 1.88e-03 & 1.65e-03 & 3.97e-03 \\ %OptTol 1e-05
$D_\text{KL}(\Proba||\PDFN)$ & 1.10e-02 & 1.11e-02 & 1.39e-02 & 1.40e-02 & 0.50e-02 & 0.62e-02 \\
\hline
\end{tabular}}
\end{center}
\end{itemize}

\vskip5pt
\begin{overprint}

\onslide<1|handout:1>
\begin{itemize}
\item PDFs:
\begin{figure}[h!]
\centering
\subfigure{\includegraphics[scale=0.2]{\figc/N3/PDF-CM}}
\subfigure{\includegraphics[scale=0.2]{\figc/N3/PDF-CDs}}
\subfigure{\includegraphics[scale=0.2]{\figc/N3/PDF-CDv}}
\end{figure}
\end{itemize}

\onslide<2|handout:2>
\begin{itemize}
\item PDFs:
\begin{figure}[h!]
\centering
\subfigure{\includegraphics[scale=0.2]{\figc/N3/PDF-AoA}}
\subfigure{\includegraphics[scale=0.2]{\figc/N3/PDF-Bend}}
\subfigure{\includegraphics[scale=0.2]{\figc/N3/PDF-Twist}}
\end{figure}
\end{itemize}

\end{overprint}

\begin{itemize}
\item Sensitivity to at most 2 or 3 parameters out of 10.
\end{itemize}

\end{frame}
%%%%%%%%%%%%%%%%%%%%%%%%%%%%%%%%%%%%%%%%%%%%%%%%%%%%%%%%%%%%%%%%%%%%%%%%%%%%%%%%%%%%%%

\section{Conclusions}

%%%%%%%%%%%%%%%%%%%%%%%%%%%%%%%%%%%%%%%%%%%%%%%%%%%%%%%%%%%%%%%%%%%%%%%%%%%%%%%%%%%%%%
%
\begin{frame}{Way forward...}
%
\begit
% 
\item Uncertainty quantification
  \begit
  \item needed for robust analysis, robust design, validation 
  \item more and more interest and projects (EU, RTO...)
  \endit
%
\vf
\item Way to proceed
  \begit
  \item Get precise definition of industry relevant problems  
  \item Use both mechanical and mathematical test cases 
  \endit
\vf
\item Challenges
  \begit
  \item Deal with large numbers of uncertain parameters
     \begit
     \item Use sensitivity analysis (Sobol indices...)
     \item Use sparsity of effects
     \endit
  \item Deal with geometrical uncertainties 
  \endit
\vf
\item Free softwares: \href{http://www.openturns.org}{\texttt{OpenTURNS}} (Airbus, EDF, Phimeca, \Onera\ $\geq 2018$), \href{https://www.uqlab.com}{\texttt{UQLab}} (ETH Z\"urich), \href{https://www.sandia.gov/UQToolkit/}{\texttt{UQTk}} (Sandia National Laboratories), \href{https://github.com/jonathf/chaospy}{\texttt{ChaosPy}} (U. of Oslo), \href{https://www.tu-chemnitz.de/etit/control/research/PoCET/}{\texttt{PoCET}} (Technische Universit\"at Chemnitz), \href{https://github.com/KTH-Nek5000/UQit}{\texttt{UQit}} (KTH), \href{https://timueh.github.io/PolyChaos.jl/stable}{\texttt{PolyChaos}} (Karlsruhe Institute of Technology)...
\endit
%
\end{frame}


%%%%%%%%%%%%%%%%%%%%%%%%%%%%%%%%%%%%%%%%%%%%%%%%%%%%%%%%%%%%%%%%%%%%%%%%%%%%%%%%%%%%%%%%%

\end{document}

%%%%%%%%%%%%%%%%%%%%%%%%%%%%%%%%%%%%%%%%%%%%%%%%%%%%%%%%%%%%%%%%%%%%%%%%%%%%%%%%%%%%%%%%%%%
